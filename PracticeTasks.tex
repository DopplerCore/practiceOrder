\sectionbreak \section*{ 
    \gostTitleFont
    \redline
    Выполненные в ходе практики задачи
}

\subtitlespace

{\gostFont
    \par \redline Преддипломная практика была пройдена в лаборатории БрГТУ <<Системы идентификации и промышленная робототехника>>. В начале практики были поставлены различные задачи, связанные не только с самим дипломным проектом, но так же и с производством ОАО <<Савушкин Продукт>>. 

    \par \redline Основной задачей преддипломной практики являлась разработка комплекса программного обеспечения для прогнозирования данных временных рядов пастеризационной установки с целью предсказания будущего поведения системы и планирования производства. Данный комплекс ПО состоит из трёх модулей: модуля для анализа, обработки и подготовки данных к прогнозированию, модуля создания, обучения и тестирования моделей нейронных сетей, а также модуля, выполняющего непосредственное прогнозирование входящих в него данных.  

    \par \redline Чтобы продолжить, необходимо разъяснить некоторые моменты, относящиеся к форматам данных. Для представления данных было разработано два формата: строчный формат OCDF и табличный формат TDF, данные которых находятся в аккуратном виде. OCDF-формат выглядит следующим образом:

    \begin{Center}
        <сид>;<время>;<значение>
    \end{Center}

    \par \redline В таком виде представлены начальные данные, которые были предосталены ОАО <<Савушкин Продукт>>. Данные формата TDF имеют следующий вид:

    \begin{Center}
        <время>;<зн1>;<зн2>;<зн3>;<зн4>;<зн5>;<зн6>
    \end{Center}

    где, знН - значение данных под сидом Н.

    \par \redline Модуль для анализа, обработки и подготовки данных к прогнозированию предоставляет ряд функций для различного преобразования данных: обрезка данных по заданному количеству, заданному проценту и до или после указанного значения дискретного времени как справа, так и слева; визуализация данных для зрительного анализа и составления инфографики; добавление новых данных к уже имеющимся для создания необходимых датасетов для моделей нейронных сетей, использующих вспомогательные данные для построение прогноза; преобразование данных в равноинтервальный временной вид, что подразумевает выравнивание взаимного расстояния между соседними точками относительно оси абсцисс; парсинг данных по указанному сиду. Все эти функции применимы к данным как OCDF-формата, так и к данным TDF-формата, за исключением последней, поскольку данные TDF-формата уже находятся в равноинтервальном временном виде.

    \par \redline Также модуль даёт возможность создания TDF-формата из данных OCDF-формата. Для выполениня данной операции необходимо сделать большое количество шагов, задействующих все возможности модуля, которые перечислены выше. 

    \par \redline Модуль создания, обучения и тестирования моделей нейронных сетей необходим для создания обученной модели нейронной сети для непосредственного прогнозирования данных как OCDF-формата, так и TDF-формата. Другими словами, благодаря этому модулю можно получить готовую к работе модель нейронной сети. Для создания такой модели предоставлены все возможности: непосредственное создание модели нейронной сети, в ходе которого имеется возможность указать её характеристики; обучение модели нейронный сети на наборах данных, подготовленных с помощью модуля для анализа, обработки и подготовки данных к прогнозированию; изменение характеристик модели нейронной сети; тестирование модели нейронной сети для настройки её гиперпараметров. 

    \par \redline Модуль для непосредственного прогнозирования данных временных рядов пастеризационной устновки говорит сам за себя о своём назначении. Этот модуль является целью данного дипломного проекта. Если два остальных модуля могут быть развёрнуты по сути где угодно, то модуль непосредственного прогнозирования как раз и должен быть развёрнут на программируемом микроконтроллере для выполения основных задач по прогнозированию. Данный модуль должен быть частью системы, состоящей из двух или более моделеей нейронной сети. Вторая часть этой системы разрабатывается моим коллегой Двораниновичем Дмитрием Александровичем. 

    \par \redline Также в ходе преддипломной практики было начато глубокое погружение в изучение \LaTeX, в ходе которого удалось приобрести большой опыт использования данных средств. Отметим наиболее важные достижения в изучение \LaTeX:

    \par \redline 1. Удалось разобраться, как подключать пользовательские шрифты. Для этого необходимо было не только скачать необходимые шрифты в формате .ttf, но также и получить метрику шрифтов в формате .tfm для возможности изменения размера символов шрифта, поскольку этот файл содержит информацию о ширине, высоте и глубине каждого символа; получить карту шрифта в формате .map для определения взаимного расположения символов; получить виртуальный шрифт в формате .vpl для его корректного отображения в pdf. Для того, чтобы получить всё вышеперечисленное необходимо было использовать консольную утилиту ttf2tfm.

    \par \redline 2. Научился создавать собственные команды.

    \par \redline 3. Создание рамок для пояснительно записки и титульного листа средствами, предоставляемыми движком \XeLaTeX. Для этого было использованы средства, позволяющие создавать графические объекты в \LaTeX.

    \par \redline 4. Подробно изучил таблицы в \LaTeX \hspace{2pt} для создания красивых разлиненных в нужном месте листов.  

    \par \redline 5. Понял, как использовать Inkscape для того, чтобы появилась возможность использовать векторные изображения в формате .svg. 

    \par \redline Также были созданы средства автосборки проекта \LaTeX, а также документация к репозиторию на GitHub, где хранится исходный код дипломного проекта на \LaTeX.

    \par
}
