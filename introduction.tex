\sectionbreak \section*{ 
    \gostTitleFont
    \redline
    ВВЕДЕНИЕ
}

\subtitlespace

{\gostFont

\par \redline В современном мире эффективность образовательного процесса во многом определяется доступностью, структурированностью и удобством использования образовательных ресурсов. Несмотря на существование различных подходов к обучению и множества образовательных платформ, наблюдается значительный разрыв между структурированным представлением информации и возможностью быстрого доступа к конкретным инструкциям, что существенно снижает эффективность образовательного процесса и затрудняет освоение новых технологий.

\par \redline Разработка образовательной платформы для обмена и изучения структурированных обучающих материалов представляет собой актуальную задачу, объединяющую современные технологии веб-разработки и инновационные подходы к организации образовательного контента. Основная цель проекта – создание системы, которая позволит пользователям не только получать доступ к структурированным обучающим материалам, но и создавать собственные образовательные ресурсы, визуализировать взаимосвязи между различными компонентами изучаемых технологий и отслеживать прогресс обучения.

\par \redline В ходе преддипломной практики на предприятии ОАО «Савушкин продукт» в бюро перспективных разработок были изучены структура предприятия, должностные инструкции работников подразделения, задачи, решаемые подразделением, используемые технические средства, а также выполнены задния по доработке существующих на предприятии проектов и разработке нового проекта.

\par
}
