\sectionbreak \section*{ 
    \gostTitleFont
    \redline
    ВВЕДЕНИЕ
}

\subtitlespace

{\gostFont

\par \redline Современная медицина всё больше полагается на технологии искусственного интеллекта и машинного обучения для повышения точности и скорости диагностики заболеваний. Одной из ключевых задач в этой области является автоматизированное распознавание патологий на основе медицинских изображений, таких как магнитно-резонансная томография (МРТ). МРТ-снимки предоставляют детальную информацию о внутренних структурах организма, что делает их незаменимыми для диагностики широкого спектра заболеваний, включая опухоли, воспаления и дегенеративные изменения. Однако анализ таких изображений требует значительных временных затрат и высокой квалификации врачей-рентгенологов. В связи с этим возникает необходимость в разработке автоматизированных систем, способных ускорить процесс диагностики и повысить её точность.

\par \redline Разработка веб-сервиса для распознавания медицинских патологий на основе МРТ-снимков представляет собой актуальную задачу, которая объединяет современные технологии машинного обучения, веб-разработки и медицинской диагностики. Основная цель проекта – создание системы, которая позволит врачам и пациентам быстро и точно анализировать медицинские изображения, выявлять патологии и получать рекомендации по дальнейшим действиям. Такой сервис не только сократит время диагностики, но и снизит нагрузку на медицинский персонал, что особенно важно в условиях растущего спроса на медицинские услуги.

\par \redline Актуальность разработки подобного веб-сервиса обусловлена несколькими факторами. Во-первых, рост количества медицинских данных, включая МРТ-снимки, требует автоматизации процессов их анализа. Во-вторых, развитие технологий машинного обучения, таких как глубокие нейронные сети, позволяет достичь высокой точности в распознавании патологий, сравнимой с экспертами-врачами. В-третьих, веб-сервисы предоставляют удобный и доступный способ взаимодействия с медицинскими данными, что делает их идеальным решением для интеграции в современные медицинские системы.

\par \redline Разрабатываемый веб-сервис станет важным инструментом для улучшения качества медицинской диагностики. Он позволит врачам быстрее и точнее выявлять патологии, что в конечном итоге приведёт к улучшению качества лечения пациентов. Кроме того, система будет способствовать снижению нагрузки на медицинский персонал и повышению эффективности работы медицинских учреждений. Внедрение таких технологий в медицинскую практику открывает новые возможности для развития персонализированной медицины и улучшения качества жизни пациентов.

\par \redline Таким образом, разработка веб-сервиса для распознавания медицинских патологий на основе МРТ-снимков является важным шагом в направлении автоматизации медицинской диагностики. Этот проект объединяет передовые технологии машинного обучения и веб-разработки, что делает его актуальным и перспективным решением для современной медицины.

\par

}
