\sectionbreak \section*{
  \gostTitleFont
  \redline
  2. ПЛАН ИНДИВИДУАЛЬНОГО ЗАДАНИЯ
}

\subsection*{
  \gostTitleFont
  \redline
  2.1 Анализ существующих решений. Описание достоинств и недостатков
}

\titlespace

\subsubsection*{ 
  \gostTitleFont
  \redline
  2.1.1 Обзор существующих подходов к обучению
} 

\subtitlespace

{\gostFont

  \par \redline В данном разделе проводится анализ существующих подходов к обучению и образовательных платформ с точки зрения эффективности образовательного процесса, доступности знаний и удобства их структурирования. Рассмотрены их функциональные возможности, преимущества и недостатки для обучающихся.

  \par \redline На сегодняшний день существует несколько основных подходов к получению образовательной информации, каждый из которых имеет свои особенности с точки зрения образовательного процесса:

  \par \redline Обучение в университете представляет собой традиционный формат получения знаний, который характеризуется следующими особенностями:
  
  \par \redline • Структурированная программа обучения с последовательным изложением материала
  \par \redline • Непосредственное взаимодействие с преподавателями и возможность получения обратной связи
  \par \redline • Практические занятия и лабораторные работы для закрепления теоретических знаний
  \par \redline • Социальное взаимодействие с другими студентами и возможность групповой работы
  \par \redline • Получение официального документа об образовании
  
  \par \redline Однако с точки зрения образовательного процесса университетское образование имеет ряд недостатков:
  
  \par \redline • Фиксированное расписание и темп обучения
  \par \redline • Ограниченная гибкость программы обучения
  \par \redline • Учебные программы часто отстают от быстро меняющихся технологий
  \par \redline • Ограниченная возможность для самостоятельного углубленного изучения отдельных тем
  \par \redline • Сложность быстрого поиска конкретной информации по решению определенной задачи

  \par \redline Онлайн-курсы (Udemy, Coursera, Pluralsight) предлагают более гибкий подход к обучению, который имеет следующие преимущества для образовательного процесса:
  
  \par \redline • Доступ к материалам в любое время и из любого места
  \par \redline • Возможность выбора конкретных курсов в зависимости от интересов и потребностей
  \par \redline • Видеоформат обучения с дополнительными материалами
  \par \redline • Система оценок и отзывов для определения качества курсов
  \par \redline • Возможность обучаться в собственном темпе, повторно просматривая сложные моменты или ускоряя просмотр знакомого материала
  
  \par \redline Однако с точки зрения образовательного процесса онлайн-курсы имеют следующие недостатки:
  
  \par \redline • Линейный формат представления информации
  \par \redline • Отсутствие быстрого доступа к необходимой информации без необходимости просмотра всего курса
  \par \redline • Ограниченное взаимодействие с преподавателем и другими обучающимися
  \par \redline • Пассивный характер обучения, не всегда обеспечивающий достаточную практику для формирования устойчивых навыков
  \par \redline • Отсутствие возможности для обучающихся создавать собственные материалы или адаптировать существующие под свои потребности

  \par \redline Обучение по книгам представляет собой традиционный метод самообразования, который имеет следующие преимущества для образовательного процесса:
  
  \par \redline • Детальное и глубокое изложение материала
  \par \redline • Возможность изучения в собственном темпе
  \par \redline • Структурированное представление информации с оглавлением и указателями
  \par \redline • Доступность без необходимости подключения к интернету
  \par \redline • Возможность делать пометки и выделять важные моменты
  
  \par \redline Однако с точки зрения образовательного процесса обучение по книгам имеет следующие недостатки:
  
  \par \redline • Ограниченные возможности для интерактивного взаимодействия с материалом
  \par \redline • Отсутствие обратной связи
  \par \redline • Сложность представления взаимосвязей между различными компонентами технологии
  \par \redline • Невозможность быстрого обновления содержания при изменении технологий
  \par \redline • Отсутствие механизмов для отслеживания прогресса обучения

  \par \redline Обучение с использованием ИИ-чатов (ChatGPT, Claude, Bard) представляет собой новый подход к получению знаний, который имеет следующие преимущества для образовательного процесса:
  
  \par \redline • Персонализированные ответы на конкретные вопросы пользователя
  \par \redline • Возможность получения информации в диалоговом формате
  \par \redline • Адаптация уровня сложности объяснений в зависимости от запросов пользователя
  \par \redline • Доступность в любое время и возможность задавать уточняющие вопросы
  \par \redline • Возможность получения информации по широкому спектру тем
  
  \par \redline Однако с точки зрения образовательного процесса обучение с использованием ИИ-чатов имеет следующие недостатки:
  
  \par \redline • Отсутствие структурированного представления информации
  \par \redline • Возможность получения неточных или устаревших данных
  \par \redline • Отсутствие визуализации взаимосвязей между компонентами технологии
  \par \redline • Сложность сохранения и организации полученных знаний
  \par \redline • Отсутствие практических заданий и проверки навыков

  \par
}

\subtitlespace

\subsubsection*{ 
  \gostTitleFont
  \redline
  2.1.2 Анализ существующих платформ для создания и обмена образовательными ресурсами
} 

\subtitlespace

{\gostFont

  \par \redline В данном разделе рассматриваются существующие платформы, которые позволяют создавать и обмениваться образовательными ресурсами, с акцентом на их влияние на эффективность образовательного процесса.

  \par \redline GitHub и GitBook представляют собой платформы для создания и хранения документации, которые имеют следующие преимущества для образовательного процесса:
  
  \par \redline • Поддержка Markdown для форматирования текста
  \par \redline • Система версионирования для отслеживания изменений
  \par \redline • Возможность совместной работы над документацией
  \par \redline • Структурированное представление информации в виде иерархии страниц
  
  \par \redline Однако с точки зрения образовательного процесса эти платформы имеют следующие недостатки:
  
  \par \redline • Ориентация на техническую документацию, а не на образовательный контент
  \par \redline • Отсутствие специализированных инструментов для создания интерактивных образовательных материалов
  \par \redline • Отсутствие механизмов для отслеживания прогресса обучения
  \par \redline • Сложность визуализации взаимосвязей между различными компонентами изучаемой технологии

  \par \redline Notion представляет собой универсальный инструмент для создания документации, баз знаний и управления проектами, который имеет следующие преимущества для образовательного процесса:
  
  \par \redline • Гибкая структура документов с возможностью вложенности
  \par \redline • Поддержка различных типов контента (текст, таблицы, списки, изображения)
  \par \redline • Возможность создания связей между страницами
  \par \redline • Шаблоны для различных типов документов
  
  \par \redline Однако с точки зрения образовательного процесса Notion имеет следующие недостатки:
  
  \par \redline • Отсутствие визуального редактора для структурирования знаний
  \par \redline • Ограниченные возможности для отслеживания прогресса обучения
  \par \redline • Отсутствие специализированных инструментов для создания пошаговых инструкций
  \par \redline • Сложность организации материалов для быстрого доступа к конкретным инструкциям без необходимости просмотра всего документа

  \par \redline Miro представляет собой онлайн-доску для совместной работы, которая имеет следующие преимущества для образовательного процесса:
  
  \par \redline • Возможность создания связей между элементами
  \par \redline • Поддержка различных типов контента (текст, изображения, видео)
  \par \redline • Совместная работа в реальном времени
  \par \redline • Шаблоны для различных типов диаграмм
  
  \par \redline Однако с точки зрения образовательного процесса Miro имеет следующие недостатки:
  
  \par \redline • Отсутствие специализации на образовательном контенте
  \par \redline • Отсутствие встроенных средств для написания технической документации с поддержкой Markdown
  \par \redline • Отсутствие механизмов для отслеживания прогресса обучения
  \par \redline • Сложность организации большого объема информации на одной доске

  \par \redline Confluence представляет собой корпоративную вики-систему для создания и хранения документации, которая имеет следующие преимущества для образовательного процесса:
  
  \par \redline • Структурированное представление информации с иерархией страниц
  \par \redline • Поддержка различных типов контента и макросов
  \par \redline • Возможность совместной работы над документацией
  \par \redline • Интеграция с другими инструментами Atlassian
  
  \par \redline Однако с точки зрения образовательного процесса Confluence имеет следующие недостатки:
  
  \par \redline • Сложный интерфейс, требующий обучения
  \par \redline • Ориентация на корпоративное использование, а не на образовательные цели
  \par \redline • Отсутствие специализированных инструментов для создания интерактивных образовательных материалов
  \par \redline • Ограниченные возможности для визуализации взаимосвязей между различными компонентами изучаемой технологии

  \par
}

\subtitlespace

\subsubsection*{ 
  \gostTitleFont
  \redline
  2.1.3 Сравнительный анализ существующих решений с точки зрения образовательного процесса
} 

\subtitlespace

{\gostFont
  
  \par \redline На основе проведенного анализа существующих подходов к обучению и платформ для создания образовательных ресурсов можно выделить следующие ключевые аспекты, влияющие на эффективность образовательного процесса:
  
  \begin{table}[h]
    \centering
    \begin{tabular}{|p{3cm}|p{6cm}|p{6cm}|}
      \hline
      \textbf{Подход/Платформа} & \textbf{Преимущества для обучения} & \textbf{Недостатки для обучения} \\
      \hline
      Университетское образование & Структурированная программа, непосредственное взаимодействие с преподавателями, практические занятия, социальное взаимодействие & Фиксированный темп обучения, ограниченная гибкость программы, устаревание материалов, сложность индивидуализации \\
      \hline
      Онлайн-курсы & Гибкий график обучения, выбор конкретных тем, обучение в собственном темпе, мультимедийный формат & Линейное представление информации, ограниченное взаимодействие, пассивный характер обучения, сложность быстрого доступа к конкретным инструкциям \\
      \hline
      Обучение по книгам & Детальное изложение материала, собственный темп изучения, структурированное представление, возможность делать пометки & Отсутствие интерактивности, нет обратной связи, статичность представления, устаревание информации, отсутствие отслеживания прогресса \\
      \hline
      ИИ-чаты & Персонализированные ответы, диалоговый формат, адаптация уровня сложности, доступность в любое время & Отсутствие структуры, возможные неточности, нет визуализации взаимосвязей, сложность сохранения знаний, отсутствие практических заданий \\
      \hline
      GitHub/GitBook & Поддержка Markdown, версионирование, совместная работа, структурированное представление & Ориентация на техническую документацию, отсутствие интерактивности, нет отслеживания прогресса, сложность визуализации взаимосвязей \\
      \hline
      Notion & Гибкая структура документов, различные типы контента, связи между страницами, шаблоны & Отсутствие визуального редактора для интерактивных материалов, ограниченное отслеживание прогресса, сложность быстрого доступа к инструкциям \\
      \hline
      Miro & Визуализация связей, различные типы контента, совместная работа, шаблоны диаграмм & Отсутствие специализации на образовании, нет поддержки Markdown, отсутствие отслеживания прогресса, сложность организации большого объема информации \\
      \hline
      Confluence & Структурированное представление, различные типы контента, совместная работа, интеграции & Сложный интерфейс, корпоративная ориентация, отсутствие интерактивности, ограниченная визуализация взаимосвязей \\
      \hline
    \end{tabular}
    \caption{Сравнение существующих решений с точки зрения образовательного процесса}
  \end{table}

  \par
}

\subtitlespace

\subsubsection*{ 
  \gostTitleFont
  \redline
  2.1.4 Выявленные проблемы существующих решений в контексте образовательного процесса
} 

\subtitlespace

{\gostFont
  
  \par \redline На основе проведенного анализа можно выделить следующие ключевые проблемы существующих решений, которые негативно влияют на эффективность образовательного процесса:
  
  \par \redline • Разрыв между структурированным представлением информации и возможностью быстрого доступа к конкретным инструкциям: традиционные образовательные форматы либо предоставляют хорошо структурированную, но трудно навигируемую информацию, либо обеспечивают быстрый доступ к отдельным фрагментам без общего контекста
  
  \par \redline • Отсутствие визуализации взаимосвязей между различными компонентами изучаемой технологии: большинство образовательных ресурсов представляют информацию в линейном формате, что затрудняет формирование системного понимания предмета и осознание взаимозависимостей между различными элементами
  
  \par \redline • Сложность баланса между детальными инструкциями и общим пониманием: существующие решения часто либо сосредоточены на детальных пошаговых инструкциях без объяснения общих принципов, либо предоставляют теоретические знания без практических руководств
  
  \par \redline • Ограниченные возможности для персонализации образовательного процесса: большинство платформ предлагают стандартизированный контент, не учитывающий индивидуальные потребности, уровень знаний и предпочтительный стиль обучения пользователя
  
  \par \redline • Недостаточная интерактивность образовательных материалов: многие ресурсы предлагают пассивное потребление информации без возможности активного взаимодействия с материалом, что снижает вовлеченность и эффективность обучения
  
  \par \redline • Отсутствие механизмов для отслеживания прогресса обучения: многие платформы не предоставляют инструментов для оценки уровня усвоения материала и отслеживания прогресса, что затрудняет планирование образовательного процесса
  
  \par \redline • Сложность создания и обмена пользовательскими образовательными ресурсами: существующие решения часто ограничивают возможности пользователей по созданию собственных материалов или их адаптации под конкретные потребности
  
  \par \redline • Отсутствие единой платформы, объединяющей преимущества различных подходов к обучению: каждое из существующих решений имеет свои сильные стороны, но ни одно не объединяет все необходимые функции для эффективного образовательного процесса

  \par
}

\subtitlespace

\subsubsection*{ 
  \gostTitleFont
  \redline
  2.1.5 Ключевые аспекты эффективного образовательного процесса
} 

\subtitlespace

{\gostFont
  
  \par \redline На основе анализа существующих решений можно выделить следующие ключевые аспекты, которые способствуют эффективному образовательному процессу и должны быть учтены при разработке новой образовательной платформы:
  
  \par \redline • Структурированное представление информации: организация учебных материалов в логической последовательности с четкой иерархией и взаимосвязями между различными элементами способствует формированию целостного понимания предмета
  
  \par \redline • Визуализация взаимосвязей: наглядное представление связей между различными компонентами изучаемой технологии помогает обучающимся формировать системное понимание и видеть общую картину
  
  \par \redline • Быстрый доступ к конкретным инструкциям: возможность быстро найти и получить доступ к пошаговым руководствам по конкретным задачам повышает практическую ценность образовательных ресурсов
  
  \par \redline • Интерактивность: активное взаимодействие с учебными материалами, возможность экспериментировать и получать обратную связь способствуют лучшему усвоению информации и формированию практических навыков
  
  \par \redline • Персонализация: адаптация образовательного процесса к индивидуальным потребностям, уровню знаний и предпочтительному стилю обучения повышает эффективность и мотивацию
  
  \par \redline • Отслеживание прогресса: возможность оценивать уровень усвоения материала и отслеживать прогресс обучения помогает планировать образовательный процесс и поддерживать мотивацию
  
  \par \redline • Создание и обмен пользовательскими ресурсами: возможность для обучающихся создавать собственные материалы, адаптировать существующие и делиться ими с другими способствует активному обучению и формированию сообщества
  
  \par \redline • Баланс между теорией и практикой: сочетание теоретических знаний с практическими руководствами обеспечивает как понимание общих принципов, так и формирование конкретных навыков
  
  \par \redline • Актуальность информации: возможность оперативного обновления учебных материалов при изменении технологий обеспечивает релевантность получаемых знаний
  
  \par \redline • Доступность и удобство использования: интуитивно понятный интерфейс, адаптивный дизайн и доступность с различных устройств снижают барьеры для обучения

  \par \redline Учет этих аспектов при разработке новой образовательной платформы позволит создать эффективный инструмент для обучения, объединяющий преимущества различных существующих подходов и устраняющий их недостатки.

  \par
}



\subsection*{
  \gostTitleFont
  \redline
  2.2 Постановка задачи и описание функций разрабатываемой системы
}

\titlespace

\subsubsection*{ 
  \gostTitleFont
  \redline
  2.2.1 Цель и задачи разработки
} 

\subtitlespace

{\gostFont

  \par \redline Основной целью дипломного проекта является разработка образовательной платформы для обмена образовательными ресурсами и их изучения, предоставляющей пользователям доступ к созданию и изучению структурированной обучающей информации по использованию различных средств, инструментов и технологий.

  \par \redline Для достижения поставленной цели необходимо решить следующие задачи:

  \par \redline • Разработать архитектуру веб-приложения с клиент-серверной архитектурой, обеспечивающую эффективное взаимодействие между клиентской и серверной частями
  \par \redline • Создать интуитивно понятный пользовательский интерфейс для работы с книгами инструкций и их элементами
  \par \redline • Реализовать функционал для создания, редактирования и визуализации страниц инструкций с использованием SVG-элементов
  \par \redline • Разработать систему создания и редактирования пошаговых инструкций с поддержкой Markdown
  \par \redline • Реализовать механизмы поиска и фильтрации образовательных ресурсов по тегам
  \par \redline • Создать систему отслеживания прогресса обучения пользователей
  \par \redline • Обеспечить возможность обмена образовательными материалами между пользователями
  \par \redline • Разработать подсистему регистрации и авторизации пользователей с использованием JWT
  \par \redline • Обеспечить безопасность и защиту данных пользователей
  \par \redline • Провести тестирование и оптимизацию разработанной системы

  \par
}

\subtitlespace

\subsubsection*{ 
  \gostTitleFont
  \redline
  2.2.2 Описание концепции образовательной платформы
} 

\subtitlespace

{\gostFont

  \par \redline Разрабатываемая образовательная платформа представляет собой веб-приложение, основанное на концепции структурированного представления обучающих материалов в виде взаимосвязанных элементов. Ключевыми компонентами платформы являются "книги инструкций", "страницы инструкций" и "элементы инструкций".

  \par \redline "Книга инструкций" представляет собой набор связанных страниц инструкций, объединенных общей тематикой или направлением обучения. Пользователи могут создавать собственные книги инструкций или находить существующие по тегам, соответствующим их интересам и образовательным потребностям.

  \par \redline "Страница инструкций" представляет собой интерактивный холст, на котором располагаются SVG-изображения, представляющие отдельные "элементы инструкций". Эти элементы могут быть связаны между собой, визуализируя рекомендуемую последовательность изучения материала или взаимосвязи между различными компонентами изучаемой технологии. Такой подход позволяет пользователю получить целостное представление о структуре изучаемой технологии или инструмента, понять взаимосвязи между отдельными компонентами и выбрать оптимальный путь обучения.

  \par \redline "Элемент инструкции" представляет собой компонент на странице инструкций, содержащий SVG-изображение и название инструкции. При нажатии на элемент инструкции открывается пошаговое описание в формате Markdown, содержащее детальное руководство по использованию определенного средства или функции. Инструкция может включать текстовые пояснения, изображения, фрагменты кода и другие элементы, необходимые для эффективного обучения.

  \par \redline Платформа предоставляет пользователям возможность не только изучать существующие обучающие материалы, но и создавать собственные. Пользователи могут разрабатывать новые книги инструкций, добавлять страницы и элементы инструкций, устанавливать связи между элементами, тем самым внося вклад в развитие образовательной базы знаний.

  \par \redline Система также включает механизмы отслеживания прогресса обучения, позволяющие пользователям контролировать свой прогресс в изучении различных инструкций. Пользователи могут отмечать изученные инструкции и добавлять книги в закладки для быстрого доступа, что помогает организовать процесс обучения более эффективно и целенаправленно.

  \par \redline В отличие от существующих альтернатив (обучение в университете, онлайн-курсы, книги, ИИ-чаты), разрабатываемая платформа объединяет преимущества различных подходов к обучению: структурированность университетского образования, гибкость онлайн-курсов, детальность книг и персонализацию ИИ-чатов, при этом устраняя их основные недостатки.

  \par
}

\subtitlespace

\subsubsection*{ 
  \gostTitleFont
  \redline
  2.2.3 Функциональные требования к системе
} 

\subtitlespace

{\gostFont

  \par \redline На основе анализа существующих решений и поставленных задач были сформулированы следующие функциональные требования к разрабатываемой системе:

  \subsubsection*{2.2.3.1 Подсистема регистрации и авторизации пользователей}

  \par \redline • Регистрация новых пользователей с использованием электронной почты и пароля
  \par \redline • Авторизация пользователей с использованием JWT-токенов
  \par \redline • Восстановление доступа к аккаунту через электронную почту
  \par \redline • Управление профилями пользователей (редактирование личной информации, изменение пароля)
  \par \redline • Разграничение прав доступа для различных категорий пользователей

  \subsubsection*{2.2.3.2 Подсистема создания и редактирования образовательных ресурсов}

  \par \redline • Создание новых книг инструкций с указанием названия и тегов
  \par \redline • Добавление, редактирование и удаление страниц инструкций в книге
  \par \redline • Создание и размещение элементов инструкций на странице с использованием SVG-изображений
  \par \redline • Установление связей между элементами инструкций для визуализации зависимостей
  \par \redline • Создание и редактирование пошаговых инструкций в формате Markdown
  \par \redline • Добавление тегов к инструкциям для улучшения поиска
  \par \redline • Предварительный просмотр создаваемых материалов
  \par \redline • Возможность копирования и адаптации существующих книг инструкций

  \subsubsection*{2.2.3.3 Подсистема поиска образовательных ресурсов}

  \par \redline • Поиск книг инструкций по названию и тегам
  \par \redline • Фильтрация результатов поиска по различным параметрам
  \par \redline • Сортировка результатов поиска по различным критериям (популярность, дата создания, рейтинг)
  \par \redline • Рекомендации похожих книг инструкций на основе истории просмотров пользователя
  \par \redline • Сохранение истории поисковых запросов пользователя
  \par \redline • Возможность добавления книг инструкций в закладки для быстрого доступа

  \subsubsection*{2.2.3.4 Подсистема отслеживания прогресса обучения}

  \par \redline • Отметка изученных инструкций
  \par \redline • Визуализация прогресса обучения по различным книгам инструкций
  \par \redline • Формирование персональных рекомендаций на основе прогресса обучения
  \par \redline • Напоминания о необходимости продолжить обучение
  \par \redline • Генерация отчетов о прогрессе обучения

  \subsubsection*{2.2.3.5 Подсистема взаимодействия пользователей}

  \par \redline • Возможность оценивания и комментирования книг инструкций
  \par \redline • Формирование рейтинга книг инструкций на основе оценок пользователей
  \par \redline • Возможность предложения исправлений и дополнений к существующим материалам
  \par \redline • Уведомления о новых комментариях и предложениях
  \par \redline • Система модерации пользовательского контента

  \par
}

\subtitlespace

\subsubsection*{ 
  \gostTitleFont
  \redline
  2.2.4 Нефункциональные требования к системе
} 

\subtitlespace

{\gostFont

  \par \redline Помимо функциональных требований, к разрабатываемой системе предъявляются следующие нефункциональные требования:

  \subsubsection*{2.2.4.1 Требования к производительности}

  \par \redline • Время отклика системы при выполнении основных операций не должно превышать 2 секунды
  \par \redline • Система должна поддерживать одновременную работу не менее 1000 пользователей
  \par \redline • Время загрузки страницы инструкций не должно превышать 3 секунды
  \par \redline • Время поиска по базе образовательных ресурсов не должно превышать 1 секунды

  \subsubsection*{2.2.4.2 Требования к безопасности}

  \par \redline • Защита персональных данных пользователей в соответствии с законодательством
  \par \redline • Шифрование паролей пользователей с использованием современных алгоритмов
  \par \redline • Использование JWT для безопасной авторизации пользователей
  \par \redline • Защита от основных типов веб-уязвимостей (XSS, CSRF, SQL-инъекции и др.)
  \par \redline • Регулярное резервное копирование данных
  \par \redline • Журналирование действий пользователей для аудита безопасности

  \subsubsection*{2.2.4.3 Требования к надежности}

  \par \redline • Доступность системы не менее 99.9\% времени
  \par \redline • Автоматическое восстановление после сбоев
  \par \redline • Сохранение данных пользователей при непредвиденном завершении сеанса
  \par \redline • Регулярное тестирование системы на наличие ошибок

  \subsubsection*{2.2.4.4 Требования к масштабируемости}

  \par \redline • Возможность горизонтального масштабирования системы при увеличении нагрузки
  \par \redline • Модульная архитектура, позволяющая добавлять новые функциональные возможности
  \par \redline • Поддержка различных типов образовательного контента

  \subsubsection*{2.2.4.5 Требования к пользовательскому интерфейсу}

  \par \redline • Интуитивно понятный интерфейс, не требующий специального обучения
  \par \redline • Адаптивный дизайн для работы на различных устройствах (компьютеры, планшеты, смартфоны)
  \par \redline • Соответствие современным стандартам веб-дизайна
  \par \redline • Поддержка различных браузеров (Chrome, Firefox, Safari, Edge)
  \par \redline • Доступность интерфейса для пользователей с ограниченными возможностями

  \par
}

\subtitlespace

\subsubsection*{ 
  \gostTitleFont
  \redline
  2.2.5 Ожидаемые результаты разработки
} 

\subtitlespace

{\gostFont

  \par \redline В результате выполнения дипломного проекта ожидается создание полнофункциональной образовательной платформы для обмена образовательными ресурсами и их изучения, предоставляющей пользователям следующие возможности:

  \par \redline • Доступ к структурированным обучающим материалам по различным технологиям и инструментам
  \par \redline • Визуальное представление взаимосвязей между компонентами изучаемых технологий в виде интерактивных страниц инструкций
  \par \redline • Пошаговые инструкции по использованию различных средств с поддержкой Markdown
  \par \redline • Создание и редактирование собственных книг инструкций
  \par \redline • Отслеживание прогресса обучения и получение персональных рекомендаций
  \par \redline • Взаимодействие с другими пользователями через комментарии и оценки
  \par \redline • Поиск и фильтрация образовательных ресурсов по тегам

  \par \redline Разработанная система должна обеспечивать удобный и эффективный процесс обучения, позволяя пользователям быстро находить необходимую информацию и структурировать свои знания. Платформа также должна способствовать созданию сообщества пользователей, заинтересованных в обмене знаниями и совместном развитии образовательных ресурсов.

  \par \redline Ожидается, что разработанная система будет востребована как индивидуальными пользователями, стремящимися освоить новые технологии и инструменты, так и образовательными учреждениями и компаниями, заинтересованными в структурированном представлении обучающих материалов для своих студентов или сотрудников.

  \par
}

\subsection*{
  \gostTitleFont
  \redline
  2.3 Выбор средств реализации
}

\titlespace

\subsubsection*{ 
  \gostTitleFont
  \redline
  2.3.1 Обоснование выбора технологий для разработки
} 

\subtitlespace

{\gostFont

  \par \redline На основе анализа требований к разрабатываемой системе и с учетом современных тенденций в веб-разработке были выбраны следующие технологии для реализации образовательной платформы. Выбор технологий осуществлялся с учетом необходимости создания масштабируемого, производительного и удобного в использовании веб-приложения с клиент-серверной архитектурой.

  \par \redline Для разработки клиентской части приложения выбрана библиотека React. Данный выбор обусловлен следующими факторами:
  
  \par \redline • Компонентный подход к разработке пользовательского интерфейса, который позволяет создавать переиспользуемые компоненты и упрощает поддержку кода
  \par \redline • Виртуальный DOM, обеспечивающий высокую производительность при обновлении интерфейса, что особенно важно при работе с интерактивными элементами страниц инструкций
  \par \redline • Большое сообщество разработчиков и обширная экосистема готовых решений, библиотек и инструментов
  \par \redline • Декларативный подход к разработке, упрощающий создание сложных пользовательских интерфейсов
  \par \redline • Возможность использования современных возможностей JavaScript (ES6+) для более эффективной разработки
  \par \redline • Хорошая интеграция с библиотеками для работы с SVG, что критически важно для реализации интерактивных страниц инструкций
  \par \redline • Поддержка однонаправленного потока данных, что упрощает отладку и тестирование приложения

  \par \redline Для разработки серверной части приложения выбран фреймворк Express.js на платформе Node.js. Данный выбор обусловлен следующими преимуществами:
  
  \par \redline • Минималистичный и гибкий фреймворк, позволяющий создавать масштабируемые веб-приложения
  \par \redline • Использование JavaScript как на клиенте, так и на сервере, что упрощает разработку и позволяет переиспользовать код
  \par \redline • Асинхронная модель обработки запросов, обеспечивающая высокую производительность при работе с большим количеством одновременных соединений
  \par \redline • Богатая экосистема middleware-компонентов, упрощающих реализацию различных функций (аутентификация, логирование, обработка ошибок и т.д.)
  \par \redline • Простая интеграция с различными базами данных, включая PostgreSQL
  \par \redline • Поддержка REST API, необходимого для взаимодействия клиентской и серверной частей приложения
  \par \redline • Возможность горизонтального масштабирования для обеспечения высокой доступности системы

  \par \redline В качестве системы управления базами данных выбрана PostgreSQL. Данный выбор обусловлен следующими факторами:
  
  \par \redline • Надежная и проверенная временем реляционная СУБД с открытым исходным кодом
  \par \redline • Поддержка сложных запросов и транзакций, необходимых для обеспечения целостности данных
  \par \redline • Возможность работы с JSON-данными, что упрощает хранение и обработку структурированных данных, таких как содержимое страниц инструкций
  \par \redline • Высокая производительность при работе с большими объемами данных
  \par \redline • Расширяемость и возможность добавления пользовательских функций и типов данных
  \par \redline • Поддержка полнотекстового поиска, что важно для реализации функционала поиска по образовательным ресурсам
  \par \redline • Хорошая интеграция с Node.js через различные ORM и драйверы

  \par \redline Для форматирования и отображения пошаговых инструкций выбран язык разметки Markdown. Данный выбор обусловлен следующими преимуществами:
  
  \par \redline • Простой и интуитивно понятный синтаксис, не требующий специальных знаний от пользователей
  \par \redline • Возможность форматирования текста, создания списков, таблиц, вставки изображений и фрагментов кода
  \par \redline • Широкая поддержка в веб-приложениях и наличие готовых библиотек для преобразования Markdown в HTML
  \par \redline • Компактность и читаемость исходного текста даже без преобразования в HTML
  \par \redline • Возможность расширения базового синтаксиса для поддержки дополнительных элементов форматирования
  \par \redline • Хорошая интеграция с системами контроля версий, что упрощает отслеживание изменений в инструкциях

  \par \redline Для реализации аутентификации и авторизации пользователей выбрана технология JSON Web Tokens (JWT). Данный выбор обусловлен следующими факторами:
  
  \par \redline • Безопасный механизм передачи информации между клиентом и сервером в виде JSON-объекта
  \par \redline • Возможность включения дополнительной информации о пользователе в токен, что уменьшает количество запросов к базе данных
  \par \redline • Отсутствие необходимости хранить состояние сессии на сервере, что упрощает масштабирование системы
  \par \redline • Поддержка механизмов цифровой подписи для обеспечения целостности данных
  \par \redline • Возможность установки срока действия токена, что повышает безопасность системы
  \par \redline • Широкая поддержка в различных языках программирования и фреймворках
  \par \redline • Простота интеграции с Express.js через middleware-компоненты

  \par
}

\subtitlespace

\subsubsection*{ 
  \gostTitleFont
  \redline
  2.3.2 Дополнительные библиотеки и инструменты
} 

\subtitlespace

{\gostFont

  \par \redline Помимо основных технологий, для реализации образовательной платформы планируется использование следующих дополнительных библиотек и инструментов:

  \subsubsection*{2.3.2.1 Библиотеки для клиентской части}

  \par \redline • React Router — библиотека для управления маршрутизацией в React-приложении, обеспечивающая навигацию между различными компонентами без перезагрузки страницы
  \par \redline • Redux — библиотека для управления состоянием приложения, обеспечивающая предсказуемое поведение и упрощающая отладку
  \par \redline • Axios — библиотека для выполнения HTTP-запросов к серверу, обеспечивающая удобный интерфейс для работы с API
  \par \redline • React-Markdown — библиотека для преобразования Markdown в React-компоненты, необходимая для отображения пошаговых инструкций
  \par \redline • React-SVG — библиотека для работы с SVG в React, обеспечивающая возможность создания и редактирования интерактивных элементов на страницах инструкций
  \par \redline • Material-UI или Ant Design — библиотека готовых компонентов пользовательского интерфейса, обеспечивающая единый стиль оформления и ускоряющая разработку
  \par \redline • React DnD (Drag and Drop) — библиотека для реализации функционала перетаскивания элементов, необходимая для удобного редактирования страниц инструкций
  \par \redline • Jest и React Testing Library — инструменты для тестирования React-компонентов, обеспечивающие качество и надежность клиентской части приложения

  \subsubsection*{2.3.2.2 Библиотеки для серверной части}

  \par \redline • Sequelize — ORM для работы с PostgreSQL, упрощающая взаимодействие с базой данных и обеспечивающая абстракцию от SQL-запросов
  \par \redline • Passport.js — middleware для аутентификации в Node.js, поддерживающий различные стратегии аутентификации, включая JWT
  \par \redline • Bcrypt — библиотека для хеширования паролей, обеспечивающая безопасное хранение учетных данных пользователей
  \par \redline • Multer — middleware для обработки multipart/form-data, необходимый для загрузки изображений и других файлов
  \par \redline • Winston или Morgan — библиотеки для логирования событий на сервере, обеспечивающие отслеживание работы приложения и диагностику ошибок
  \par \redline • Joi или Express-validator — библиотеки для валидации данных, обеспечивающие проверку входных данных на соответствие требованиям
  \par \redline • Nodemailer — библиотека для отправки электронных писем, необходимая для функционала восстановления пароля и уведомлений
  \par \redline • Socket.io — библиотека для реализации двунаправленной связи между клиентом и сервером в реальном времени, необходимая для функционала совместной работы над образовательными ресурсами

  \subsubsection*{2.3.2.3 Инструменты разработки и развертывания}

  \par \redline • Webpack — инструмент для сборки и оптимизации клиентской части приложения, обеспечивающий минификацию кода и управление зависимостями
  \par \redline • Babel — транспилятор JavaScript, позволяющий использовать современные возможности языка с сохранением совместимости с различными браузерами
  \par \redline • ESLint — инструмент для статического анализа кода, обеспечивающий соблюдение стандартов кодирования и выявление потенциальных ошибок
  \par \redline • Prettier — инструмент для автоматического форматирования кода, обеспечивающий единый стиль оформления
  \par \redline • Git — система контроля версий, обеспечивающая отслеживание изменений в коде и совместную работу над проектом
  \par \redline • Docker — платформа для контейнеризации приложений, обеспечивающая единообразие среды разработки и развертывания
  \par \redline • CI/CD (например, GitHub Actions или GitLab CI) — инструменты для непрерывной интеграции и доставки, автоматизирующие процессы тестирования и развертывания приложения

  \par
}

\subtitlespace

\subsubsection*{ 
  \gostTitleFont
  \redline
  2.3.3 Архитектурные решения и паттерны проектирования
} 

\subtitlespace

{\gostFont

  \par \redline Для обеспечения масштабируемости, поддерживаемости и гибкости разрабатываемой системы планируется использование следующих архитектурных решений и паттернов проектирования:

  \subsubsection*{2.3.3.1 Клиентская часть}

  \par \redline • Компонентная архитектура — разделение пользовательского интерфейса на независимые, переиспользуемые компоненты, каждый из которых отвечает за определенную функциональность
  \par \redline • Flux-архитектура (через Redux) — однонаправленный поток данных, обеспечивающий предсказуемое изменение состояния приложения и упрощающий отладку
  \par \redline • Container/Presentational паттерн — разделение компонентов на контейнеры, отвечающие за логику и получение данных, и презентационные компоненты, отвечающие только за отображение
  \par \redline • Render Props и Higher-Order Components — паттерны для переиспользования логики между компонентами
  \par \redline • Lazy Loading — отложенная загрузка компонентов и ресурсов для улучшения производительности приложения
  \par \redline • Мемоизация — оптимизация производительности путем кеширования результатов вычислений

  \subsubsection*{2.3.3.2 Серверная часть}

  \par \redline • MVC (Model-View-Controller) — разделение серверной логики на модели (работа с данными), представления (формирование ответов) и контроллеры (обработка запросов)
  \par \redline • Repository Pattern — абстракция доступа к данным, скрывающая детали работы с базой данных и упрощающая тестирование
  \par \redline • Middleware Pattern — цепочка обработчиков запросов, позволяющая модульно добавлять функциональность (аутентификация, логирование, обработка ошибок)
  \par \redline • Dependency Injection — внедрение зависимостей для уменьшения связанности компонентов и упрощения тестирования
  \par \redline • Service Layer — выделение бизнес-логики в отдельный слой, независимый от контроллеров и моделей
  \par \redline • Error Handling Middleware — централизованная обработка ошибок для обеспечения единообразного поведения приложения при возникновении исключений

  \subsubsection*{2.3.3.3 Общие архитектурные решения}

  \par \redline • RESTful API — архитектурный стиль взаимодействия между клиентом и сервером, основанный на принципах REST
  \par \redline • JWT-based Authentication — механизм аутентификации на основе токенов, не требующий хранения состояния сессии на сервере
  \par \redline • Microservices (при необходимости масштабирования) — разделение приложения на независимые сервисы, каждый из которых отвечает за определенную функциональность
  \par \redline • CQRS (Command Query Responsibility Segregation) — разделение операций чтения и записи для оптимизации производительности и масштабируемости
  \par \redline • Event-driven Architecture — использование событий для обеспечения слабой связанности компонентов системы
  \par \redline • Caching Strategies — стратегии кеширования данных для улучшения производительности и уменьшения нагрузки на базу данных

  \par \redline Выбранные архитектурные решения и паттерны проектирования обеспечат создание гибкой, масштабируемой и поддерживаемой системы, способной эффективно решать поставленные задачи и адаптироваться к изменяющимся требованиям.

  \par
}

\subtitlespace

\subsubsection*{ 
  \gostTitleFont
  \redline
  2.3.4 Обоснование выбора структуры базы данных
} 

\subtitlespace

{\gostFont

  \par \redline Для хранения данных образовательной платформы выбрана реляционная модель базы данных, реализуемая с помощью PostgreSQL. Данный выбор обусловлен необходимостью обеспечения целостности данных, поддержки сложных связей между сущностями и возможности выполнения комплексных запросов.

  \par \redline Основными сущностями базы данных являются:

  \par \redline • Пользователь (User) — хранит информацию о пользователях системы, включая учетные данные, контактную информацию и настройки профиля. Атрибуты: id, name, email, password, learnedInstructionsIds (массив идентификаторов изученных инструкций), bookmarkedBooksIds (массив идентификаторов книг в закладках).
  
  \par \redline • Книга инструкций (Book) — представляет собой набор связанных страниц инструкций, объединенных общей тематикой. Атрибуты: id, name, ownerId (идентификатор пользователя-владельца), а также дополнительные атрибуты, такие как description (описание книги), createdAt (дата создания), updatedAt (дата обновления), tags (теги для поиска).
  
  \par \redline • Страница инструкций (Page) — содержит информацию о странице с инструкциями, включая ее структуру и содержимое. Атрибуты: id, bookId (идентификатор книги, к которой относится страница), title (название страницы), instructions (массив или JSON-структура, описывающая элементы инструкций на странице), order (порядковый номер страницы в книге).
  
  \par \redline • Инструкция (Instruction) — содержит детальную информацию об отдельной инструкции, включая ее содержимое и метаданные. Атрибуты: id, name (название инструкции), position (координаты расположения на странице), dependencies (массив идентификаторов инструкций, от которых зависит данная инструкция), markdownText (текст инструкции в формате Markdown), tags (теги для поиска).

  \par \redline Дополнительные сущности, необходимые для реализации всех требуемых функций системы:

  \par \redline • Комментарий (Comment) — хранит комментарии пользователей к книгам и инструкциям. Атрибуты: id, userId (идентификатор автора комментария), targetType (тип объекта комментирования: книга или инструкция), targetId (идентификатор объекта комментирования), text (текст комментария), createdAt (дата создания).
  
  \par \redline • Оценка (Rating) — хранит оценки пользователей для книг инструкций. Атрибуты: id, userId (идентификатор пользователя), bookId (идентификатор книги), value (значение оценки), createdAt (дата создания).
  
  \par \redline • Тег (Tag) — хранит информацию о тегах, используемых для категоризации и поиска образовательных ресурсов. Атрибуты: id, name (название тега), description (описание тега).
  
  \par \redline • Прогресс обучения (LearningProgress) — отслеживает прогресс пользователей в изучении различных книг и инструкций. Атрибуты: id, userId (идентификатор пользователя), instructionId (идентификатор инструкции), status (статус изучения), completedAt (дата завершения изучения).

  \par \redline Для обеспечения эффективной работы с данными и оптимизации запросов планируется использование следующих подходов:

  \par \redline • Индексирование ключевых полей для ускорения поиска и сортировки
  \par \redline • Использование внешних ключей для обеспечения ссылочной целостности данных
  \par \redline • Нормализация данных для минимизации избыточности и аномалий обновления
  \par \redline • Использование JSON-типов данных для хранения структурированной информации, такой как позиции элементов на странице инструкций
  \par \redline • Применение транзакций для обеспечения атомарности операций, затрагивающих несколько таблиц
  \par \redline • Использование полнотекстового поиска PostgreSQL для эффективного поиска по содержимому инструкций

  \par \redline Выбранная структура базы данных обеспечивает гибкость, масштабируемость и производительность, необходимые для реализации всех функциональных требований к образовательной платформе, а также позволяет легко расширять систему новыми возможностями в будущем.

  \par
}

\subtitlespace

\subsubsection*{ 
  \gostTitleFont
  \redline
  2.3.5 Выбор инструментов для обеспечения безопасности
} 

\subtitlespace

{\gostFont

  \par \redline Безопасность является критически важным аспектом разрабатываемой образовательной платформы, особенно учитывая необходимость защиты персональных данных пользователей и образовательного контента. Для обеспечения безопасности системы выбраны следующие инструменты и подходы:

  \subsubsection*{2.3.5.1 Аутентификация и авторизация}

  \par \redline • JSON Web Tokens (JWT) — для безопасной передачи информации о пользователе между клиентом и сервером без необходимости хранения состояния сессии на сервере
  \par \redline • Bcrypt — для хеширования паролей пользователей с использованием соли и алгоритма, устойчивого к атакам перебором
  \par \redline • Passport.js — для реализации различных стратегий аутентификации и управления процессом авторизации
  \par \redline • HTTPS — для шифрования данных, передаваемых между клиентом и сервером, с использованием TLS/SSL сертификатов
  \par \redline • CORS (Cross-Origin Resource Sharing) — для контроля доступа к ресурсам сервера с других доменов

  \subsubsection*{2.3.5.2 Защита от распространенных уязвимостей}

  \par \redline • Helmet.js — для установки HTTP-заголовков безопасности, защищающих от различных типов атак (XSS, clickjacking, MIME-sniffing и др.)
  \par \redline • Express-validator или Joi — для валидации и санитизации входных данных, предотвращающих инъекции и другие атаки, связанные с некорректными данными
  \par \redline • CSRF-токены — для защиты от Cross-Site Request Forgery атак
  \par \redline • Content Security Policy (CSP) — для ограничения источников загружаемых ресурсов и предотвращения XSS-атак
  \par \redline • Rate Limiting — для защиты от DDoS-атак и брутфорс-атак на систему аутентификации

  \subsubsection*{2.3.5.3 Безопасность данных}

  \par \redline • Параметризованные запросы — для предотвращения SQL-инъекций при работе с базой данных
  \par \redline • Шифрование чувствительных данных в базе данных — для защиты персональной информации пользователей
  \par \redline • Регулярное резервное копирование данных — для обеспечения возможности восстановления в случае потери или повреждения данных
  \par \redline • Управление доступом на уровне базы данных — для ограничения прав доступа к данным в соответствии с ролями пользователей

  \subsubsection*{2.3.5.4 Мониторинг и аудит безопасности}

  \par \redline • Winston или Morgan — для логирования событий безопасности и действий пользователей
  \par \redline • Sentry или аналогичные сервисы — для отслеживания ошибок и потенциальных проблем безопасности в реальном времени
  \par \redline • Регулярное сканирование уязвимостей — для выявления и устранения потенциальных проблем безопасности
  \par \redline • Dependency scanning — для проверки используемых библиотек на наличие известных уязвимостей

  \par \redline Выбранные инструменты и подходы к обеспечению безопасности позволят создать надежную и защищенную образовательную платформу, соответствующую современным стандартам безопасности веб-приложений и требованиям законодательства в области защиты персональных данных.

  \par
}

\subtitlespace

\subsubsection*{ 
  \gostTitleFont
  \redline
  2.3.6 Обоснование выбора инструментов для разработки пользовательского интерфейса
} 

\subtitlespace

{\gostFont

  \par \redline Пользовательский интерфейс является критически важным компонентом образовательной платформы, так как от его удобства и интуитивности напрямую зависит эффективность обучения и удовлетворенность пользователей. Для разработки пользовательского интерфейса выбраны следующие инструменты и подходы:

  \subsubsection*{2.3.6.1 Основные инструменты для создания интерфейса}

  \par \redline • React — библиотека для создания пользовательских интерфейсов, выбранная в качестве основы клиентской части приложения
  \par \redline • Material-UI или Ant Design — библиотека готовых компонентов, обеспечивающая единый стиль оформления и соответствие современным стандартам веб-дизайна
  \par \redline • CSS Modules или Styled Components — для стилизации компонентов с изоляцией стилей и предотвращением конфликтов имен классов
  \par \redline • Flexbox и CSS Grid — для создания адаптивных и гибких макетов страниц
  \par \redline • SVG — для создания и отображения векторной графики на страницах инструкций

  \subsubsection*{2.3.6.2 Инструменты для работы с SVG и интерактивными элементами}

  \par \redline • React-SVG — библиотека для работы с SVG в React, обеспечивающая возможность создания и редактирования интерактивных элементов
  \par \redline • D3.js — библиотека для создания сложных визуализаций и интерактивных графических элементов
  \par \redline • React DnD (Drag and Drop) — библиотека для реализации функционала перетаскивания элементов, необходимая для удобного редактирования страниц инструкций
  \par \redline • React-Konva — библиотека для работы с Canvas в React, предоставляющая альтернативный способ создания интерактивных графических элементов

  \subsubsection*{2.3.6.3 Инструменты для работы с Markdown}

  \par \redline • React-Markdown — библиотека для преобразования Markdown в React-компоненты, необходимая для отображения пошаговых инструкций
  \par \redline • Remark и Rehype — плагины для расширения возможностей Markdown, такие как поддержка синтаксиса для выделения кода, таблиц, математических формул и т.д.
  \par \redline • CodeMirror или Monaco Editor — редакторы кода для создания и редактирования Markdown-контента с подсветкой синтаксиса и предварительным просмотром

  \subsubsection*{2.3.6.4 Инструменты для улучшения пользовательского опыта}

  \par \redline • React Router — библиотека для управления маршрутизацией в React-приложении, обеспечивающая навигацию между различными компонентами без перезагрузки страницы
  \par \redline • React Query или SWR — библиотеки для управления состоянием данных, получаемых с сервера, обеспечивающие кеширование, автоматическое обновление и обработку ошибок
  \par
