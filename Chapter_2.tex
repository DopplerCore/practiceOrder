\sectionbreak \section*{
  \cyrillicfont 
  \fontsize{14pt}{0pt}\selectfont
  \englishfont 
  \redline
  2. АНАЛИЗ СУЩЕСТВУЮЩИХ РЕШЕНИЙ
}

\titlespace

\subsection*{ 
  \gostTitleFont
  \redline
  2.1 Анализ существующих решений. Описание достоинств и недостатков 
} 

\subtitlespace

{\gostFont

  \par \redline Веб-сервис для распознавания медицинских патологий на основе МРТ-снимков представляет собой сложную систему, объединяющую медицинскую диагностику, искусственный интеллект и веб-технологии. Основная цель проекта — предоставить врачам и пациентам инструмент для автоматизированного анализа медицинских изображений, что позволит ускорить процесс диагностики и повысить её точность. В рамках данной главы проводится анализ существующих решений в области автоматизированной диагностики медицинских изображений, рассматриваются их достоинства и недостатки, а также выявляются ключевые проблемы, которые необходимо решить в рамках данного проекта.

  \par \redline В настоящее время существует множество приложений и систем, связанных с автоматизированной диагностикой медицинских изображений, таких как МРТ, КТ и рентген. Наиболее популярные из них включают Google DeepMind Health, IBM Watson Health, Aidoc, Zebra Medical Vision, ITK-SNAP, 3D Slicer, U-Net, CheXpert и MONAI. Эти решения предлагают различные подходы к анализу медицинских изображений, начиная от классических алгоритмов обработки изображений и заканчивая современными методами машинного и глубокого обучения.

  \par \redline Google DeepMind Health – это платформа, которая использует глубокое обучение для анализа медицинских изображений, включая диагностику заболеваний глаз и рака. Основное преимущество DeepMind Health – высокая точность диагностики, сравнимая с экспертами-врачами. Однако система требует больших объёмов данных для обучения и значительных вычислительных ресурсов. Кроме того, интерпретируемость модели остаётся проблемой, так как врачи не всегда могут понять, на основе каких признаков модель делает выводы.

  \par \redline IBM Watson Health – это платформа, которая применяет машинное обучение для анализа медицинских данных, включая изображения. Основное достоинство платформы – её универсальность: она может работать с различными типами данных, включая медицинские изображения, текстовые отчёты и генетическую информацию. Однако, как и в случае с DeepMind, Watson Health требует больших объёмов данных для обучения, а её использование может быть дорогостоящим для небольших клиник.

  \par \redline Aidoc – это коммерческая платформа для анализа медицинских изображений на основе искусственного интеллекта. Aidoc интегрируется с системами PACS (Picture Archiving and Communication System) и помогает врачам в диагностике, автоматически выделяя патологии на снимках. Основное преимущество Aidoc – её готовность к использованию: платформа не требует глубоких знаний в области машинного обучения и легко интегрируется с существующими медицинскими системами. Однако Aidoc ограничена функционалом, предоставляемым разработчиками, и пользователи не могут легко модифицировать алгоритмы.

  \par \redline Zebra Medical Vision – это компания, предлагающая решения для автоматизированного анализа медицинских изображений, включая диагностику рака, сердечно-сосудистых заболеваний и других патологий. Zebra Medical Vision использует глубокое обучение для анализа снимков и предоставляет врачам подробные отчёты. Основное достоинство платформы – её высокая точность и возможность адаптации для различных типов патологий. Однако, как и другие коммерческие решения, Zebra Medical Vision может быть дорогостоящим для небольших клиник, а её использование требует интеграции с существующими системами.

  \par \redline ITK-SNAP – это программа для сегментации медицинских изображений, которая использует классические алгоритмы для выделения областей интереса. ITK-SNAP позволяет врачам вручную настраивать параметры сегментации, что делает её полезной для исследовательских задач. Основное преимущество ITK-SNAP – её интерпретируемость: врачи могут чётко понять, какие именно признаки были использованы для диагностики. Однако программа ограничена в точности и не может автоматически улучшаться с увеличением объёма данных.

  \par \redline 3D Slicer – это инструмент для анализа медицинских изображений, который включает в себя множество классических методов обработки. 3D Slicer позволяет врачам визуализировать и анализировать трёхмерные медицинские изображения, что делает её полезной для планирования хирургических операций. Основное достоинство 3D Slicer – её универсальность: программа поддерживает различные форматы изображений и может быть адаптирована для различных задач. Однако, как и ITK-SNAP, 3D Slicer требует ручной настройки и не может автоматически обучаться на данных.

  \par \redline U-Net – это архитектура свёрточной нейронной сети, разработанная для биомедицинской сегментации изображений. U-Net широко используется для анализа МРТ и КТ, показывая высокую точность в задачах сегментации. Основное преимущество U-Net – её универсальность: одна и та же архитектура может быть адаптирована для различных типов патологий и модальностей изображений. Однако U-Net требует больших объёмов данных для обучения и значительных вычислительных ресурсов.

  \par \redline CheXpert – это модель на основе глубокого обучения для анализа рентгеновских снимков грудной клетки. CheXpert позволяет врачам быстро и точно диагностировать заболевания лёгких, такие как пневмония и рак. Основное достоинство CheXpert – её высокая точность и возможность автоматического обучения на данных. Однако модель требует больших объёмов размеченных данных и может быть чувствительна к шуму и артефактам на снимках.

  \par \redline MONAI – это платформа для разработки и внедрения моделей глубокого обучения в медицинской визуализации. MONAI предоставляет инструменты для создания и обучения моделей, что делает её полезной для исследователей и разработчиков. Основное преимущество MONAI – её гибкость: платформа поддерживает различные архитектуры нейронных сетей и может быть адаптирована для различных задач. Однако MONAI требует глубоких знаний в области машинного обучения и не является готовым решением для клинического использования.

  \par \redline Каждое из этих приложений и систем имеет свои достоинства и недостатки. Основные различия заключаются в подходе к анализу данных (классические методы, машинное обучение, глубокое обучение), функциональности (готовые решения vs. исследовательские инструменты) и требованиях к данным и вычислительным ресурсам. Однако все они направлены на одну цель – улучшение процесса диагностики и повышение точности анализа медицинских изображений.

  \par \redline В отличие от существующих решений, предлагаемый веб-сервис может агрегировать в себе несколько моделей машинного обучения для распознавания медицинских патологий пациента, выбирая наиболее оптимальные подходы в зависимости от типа данных и поставленной задачи. Это позволяет повысить точность диагностики, адаптироваться к различным типам патологий и минимизировать ограничения, связанные с использованием单一 моделей.
  
  \par
}

\subtitlespace

\subsection*{ 
  \gostTitleFont
  \redline
  2.2 Постановка задачи и описание функций разрабатываемой системы 
} 

\subtitlespace

{\gostFont

  \par \redline Веб-сервис для распознавания медицинских патологий на основе МРТ-снимков представляет собой сложную систему, объединяющую медицинскую диагностику, искусственный интеллект и веб-технологии. Основная цель проекта — предоставить врачам и пациентам инструмент для автоматизированного анализа медицинских изображений, что позволит ускорить процесс диагностики и повысить её точность. В рамках данной главы проводится анализ существующих решений в области автоматизированной диагностики медицинских изображений, рассматриваются их достоинства и недостатки, а также выявляются ключевые проблемы, которые необходимо решить в рамках данного проекта.

  \par
}

\subtitlespace

\subsection*{ 
  \gostTitleFont
  \redline
  2.3 Выбор средств реализации 
} 

\subtitlespace

{\gostFont

  
}
