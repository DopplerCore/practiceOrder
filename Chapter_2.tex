\sectionbreak \section*{
  \gostTitleFont
  \redline
  2. ПЛАН ИНДИВИДУАЛЬНОГО ЗАДАНИЯ
}

\titlespace

\subsection*{
  \gostTitleFont
  \redline
  2.1 Анализ существующих решений. Описание достоинств и недостатков
}

{\gostFont

  \par \redline В данном разделе проводится анализ существующих подходов и платформ для диагностики заболеваний животных, с акцентом на их точность, доступность и удобство использования. Рассмотрены их функциональные возможности, преимущества и недостатки для фермеров и ветеринаров.

  \par \redline На сегодняшний день существует несколько основных подходов к диагностике заболеваний животных, каждый из которых имеет свои особенности:
  
  \par \redline Традиционная ветеринарная диагностика

  \par \redline Традиционная диагностика проводится ветеринарными специалистами в клиниках или на дому с использованием визуального осмотра, лабораторных исследований и инструментальной диагностики. Ветеринар проводит физический осмотр животного, оценивая его общее состояние, поведение и внешние признаки заболеваний. При необходимости берутся образцы крови, мочи, кала или тканей для лабораторного анализа. На основе полученных данных и своего опыта ветеринар ставит диагноз и назначает лечение. Такой подход имеет следующие преимущества:
  
  \par \redline • Высокая точность диагностики благодаря опыту специалиста.
  \par \redline • Возможность комплексного обследования животного.
  \par \redline • Использование лабораторных методов для подтверждения диагноза.

  \par \redline При всех своих преимуществах существуют и недостатки.

  \par \redline • Высокая стоимость услуг ветеринара.
  \par \redline • Необходимость физического присутствия специалиста.
  \par \redline • Длительное время ожидания результатов анализов.
  \par \redline • Ограниченная доступность ветеринарных услуг в удаленных регионах.


  \par \redline Анализ звуковых данных

  \par \redline Данный подход основан на обработке звуковых сигналов, таких как дыхание, кашель или вокализация животных. Используются рекуррентные нейросети (RNN, LSTM) и алгоритмы спектрального анализа звука. Преимущества такого подохода.
  
  \par \redline • Позволяет диагностировать респираторные заболевания без необходимости визуального осмотра;
  \par \redline • Может применяться для раннего выявления проблем со здоровьем, когда внешние признаки отсутствуют;
  \par \redline • Возможность интеграции с носимыми устройствами для мониторинга.

  \par \redline Недостатки.
  
  \par \redline • Высокая чувствительность к шумам окружающей среды;
  \par \redline • Требует обучения на большом количестве аудиоданных;
  \par \redline • Ограниченная применимость для некоторых типов заболеваний.


  \par \redline Анализ поведения животных

  \par \redline Этот метод использует видеонаблюдение и алгоритмы машинного обучения для выявления изменений в активности животных. Технологии Pose Estimation (например, OpenPose, DeepLabCut) помогают анализировать позы, движения и привычки животных. Преимущества такого подохода.
  
  \par \redline • Может выявлять невидимые внешние симптомы (например, снижение активности, вялость, стресс);
  \par \redline • Не требует непосредственного контакта с животным;
  \par \redline • Может использоваться для мониторинга в режиме реального времени.

  \par \redline Недостатки.
  
  \par \redline • Требует постоянного видеонаблюдения;
  \par \redline • Высокая вычислительная сложность анализа видеоданных;
  \par \redline • Возможность ложных срабатываний при изменении окружающей среды.


  \par \redline Биометрический анализ с помощью сенсоров

  \par \redline Использование умных ошейников и других носимых устройств для измерения температуры тела, сердечного ритма и других показателей. Алгоритмы машинного обучения анализируют отклонения от нормы и предсказывают возможные заболевания. Преимущества такого подохода.
  
  \par \redline • Возможность постоянного мониторинга состояния животного;
  \par \redline • Раннее выявление отклонений, до появления внешних симптомов;
  \par \redline • Удобство для владельцев домашних животных и фермеров.

  \par \redline Недостатки.
  
  \par \redline • Высокая стоимость сенсорных устройств;
  \par \redline • Необходимость регулярного обслуживания оборудования;
  \par \redline • Возможны погрешности при сборе данных.
  
  
  \par \redline Существующие платформы для диагностировая животных

  \par \redline VetExpert – это онлайн-платформа для ветеринаров и владельцев домашних животных, предоставляющая справочную информацию о заболеваниях, их симптомах и методах лечения. Преимущества данной платформы.
  
  \par \redline • Большая база данных по заболеваниям различных видов животных;
  \par \redline • Удобный интерфейс для поиска информации по симптомам;
  \par \redline • Доступен бесплатно для пользователей.

  \par \redline Недостатки.
  
  \par \redline • Не проводит автоматическую диагностику;
  \par \redline • Не учитывает индивидуальные особенности конкретного животного;
  \par \redline • Требует самостоятельного анализа информации пользователем.

  \par \redline WebMD Pet Health Center – это подразделение известного медицинского ресурса WebMD, ориентированное на здоровье домашних животных. Оно предоставляет статьи и рекомендации по уходу, а также симптомы возможных заболеваний. Преимущества данной платформы.
  
  \par \redline • Надежный источник информации, основанный на ветеринарных исследованиях;
  \par \redline • Простота использования для владельцев животных;
  \par \redline • Бесплатный доступ к материалам.

  \par \redline Недостатки.
  
  \par \redline • Отсутствие персонализированной диагностики;
  \par \redline • Не учитывает поведенческие факторы и внешние признаки заболеваний;
  \par \redline • Нет интеграции с ветеринарными клиниками.


  \par \redline OneMind Dogs использует машинное обучение и анализ поведения для выявления отклонений в активности животных, что может свидетельствовать о стрессе или заболеваниях. Преимущества данной платформы.
  
  \par \redline • Анализирует движения и поведение животных для диагностики психологических и физических отклонений;
  \par \redline • Интерактивные рекомендации по уходу за питомцем;
  \par \redline • Доступен для владельцев домашних животных без необходимости медицинского оборудования.

  \par \redline Недостатки.
  
  \par \redline • Не заменяет полноценную медицинскую диагностику;
  \par \redline • Может давать ложноположительные результаты при изменении среды;
  \par \redline • Требует длительного периода наблюдения для сбора данных.

  
  \par \redline На основе проведенного анализа можно выделить следующие ключевые проблемы существующих решений, которые негативно влияют на эффективность диагностики заболеваний у животных:
  
  \par \redline • Ограниченная точность диагностики без профессионального участия. Большинство сервисов, использующих нейросети или справочные базы, не могут полностью заменить консультацию ветеринара и зачастую дают ложноположительные или ложноотрицательные результаты.
  \par \redline • Отсутствие единого подхода к диагностике. Разные платформы специализируются на отдельных аспектах (анализ изображений, симптомы, токсикология и т. д.), но нет единого сервиса, который бы комплексно оценивал состояние животного.
  \par \redline • Сложность интерпретации результатов для владельцев животных. Многие платформы предоставляют либо слишком технические данные, понятные только специалистам, либо наоборот – слишком обобщенные рекомендации, которые не всегда полезны для точного определения состояния питомца.
  \par \redline • Необходимость ручного ввода данных. Многие платформы требуют от пользователя самостоятельного поиска симптомов и их ввода, что увеличивает вероятность ошибок и субъективного восприятия.
  \par \redline • Ограниченный доступ к технологиям. Некоторые платформы доступны только в определенных странах или требуют специализированного оборудования (например, рентген-аппарата), что делает их недоступными для широкого круга пользователей.
  \par \redline • Отсутствие интеграции с ветеринарными клиниками. Сервисы, предлагающие диагностику, редко взаимодействуют с ветеринарными учреждениями, что затрудняет дальнейшее лечение животного на основе полученных данных.

  \par
}
\subtitlespace


\subsection*{
  \gostTitleFont
  \redline
  2.2 Постановка задачи и описание функций разрабатываемой системы
}

\subtitlespace

{\gostFont

  \par \redline Основной целью дипломного проекта является разработка веб-платформы для диагностики заболеваний животных с использованием нейронных сетей. Платформа должна предоставлять пользователям (фермерам и ветеринарам) возможность быстро и точно определять наличие заболеваний у животных на основе введенных симптомов и данных. Платформа также должна быть простой в использовании и доступной для широкого круга пользователей.

  \par \redline Для достижения поставленной цели необходимо решить следующие задачи:

  \par \redline • Разработать архитектуру веб-приложения с клиент-серверной архитектурой, обеспечивающую эффективное взаимодействие между клиентской и серверной частями
  \par \redline • Создать интуитивно понятный пользовательский интерфейс, для ввода данных о состоянии животного (симптомы, поведение, показатели здоровья) и обеспечить визуализацию результатов диагностики
  \par \redline • Сбор набора данных
  \par \redline • Обработка набора данных и анализ с помощью нейронных сетей для классификации состояния здоровья животного
  \par \redline • Обучение нейронной сети на основе набора данных, чтобы она могла распознавать признаки заболеваний.
  \par \redline • Разработка интерфейса для ввода изображений и получения результатов диагностики
  \par \redline • Возможность обновления модели при добавлении новых данных
  \par \redline • Разработать систему создания и редактирования базы данных заболеваний
  \par \redline • Создать систему отслеживания прогресса диагностики
  \par \redline • Разработать подсистему регистрации и авторизации пользователей
  \par \redline • Обеспечить безопасность и защиту данных пользователей
  \par \redline • Провести тестирование и оптимизацию разработанной системы

  \par
}

\subtitlespace

\subsection*{
  \gostTitleFont
  \redline
  2.3 Выбор средств реализации
}

\subtitlespace

{\gostFont

  \par \redline  Для реализации системы был выбран язык программирования Python, который обладает кроссплатформенностью, динамической типизацией и широким набором библиотек для различных задач. Python позволяет создавать программы, которые компилируются в байт-код и исполняются на виртуальной машине, что обеспечивает их работу на различных платформах. Кроме того, Python является легким в освоении языком, предоставляющим множество библиотек, упрощающих разработку программного обеспечения.

  \par \redline Библиотеки для машинного обучения

  \par \redline В Python существует множество фреймворков для машинного обучения, таких как TensorFlow, PyTorch и Keras. В рамках данного проекта будет использоваться PyTorch, который предоставляет удобный высокоуровневый API для создания нейронных сетей. PyTorch был выбран благодаря своей гибкости, поддержке динамических вычислений и активному сообществу разработчиков. Также будет использоваться библиотека SHAP для определения наиболее влиятельных параметров в работе нейронной сети и визуализации результатов с использованием библиотеки Matplotlib.
  \par \redline Фреймворк для создания веб-приложения
  \par \redline Для создания веб-приложения используется фреймворк Flask, который был выбран благодаря своей простоте и минимальным требованиям к настройке. Flask позволяет быстро создавать REST API и интегрировать его с другими компонентами системы. Для разработки клиентской части сайта используется JavaScript вместе с библиотекой React.js, которая обеспечивает создание динамического и интерактивного пользовательского интерфейса.
  \par \redline Для обеспечения безопасности данных используются следующие библиотеки:
  \par \redline • PyCryptoDome: предоставляет необходимые алгоритмы шифрования, такие как RSA, AES и другие. Используя асинхронный алгоритм шифрования RSA, можно публиковать публичный ключ в открытый доступ и расшифровывать полученные данные с помощью закрытого ключа, не рискуя безопасностью системы.
  \par \redline • PyJWT: используется для реализации токенов доступа. Библиотека позволяет шифровать токены, автоматически проверять их расшифровку и время истечения, что делает её удобным инструментом для создания надёжных токенов доступа.
  \par \redline База данных
  \par \redline Для хранения данных используется реляционная база данных PostgreSQL, которая обеспечивает высокую производительность и надежность. Для работы с базой данных используется библиотека SQLAlchemy, которая предоставляет удобный интерфейс для создания запросов CRUD (create, read, update, delete). Для реализации асинхронного взаимодействия с базой данных используется движок asyncpg, который обеспечивает высокую производительность и эффективность при работе с PostgreSQL в асинхронном режиме.
  \par \redline Дополнительные инструменты
  \par \redline • Pandas: для обработки и анализа табличных данных.
  \par \redline • NumPy: для выполнения математических операций и работы с массивами данных.
  \par \redline • Bootstrap: для создания адаптивного и современного пользовательского интерфейса.
  \par \redline • Pytest: для тестирования кода и обеспечения его качества.



  \par
}
