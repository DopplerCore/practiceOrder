\sectionbreak \section*{
  \gostTitleFont
  \redline
  1. ОБЩАЯ ХАРАКТЕРИСТИКА ПРЕДПРИЯТИЯ
}

\titlespace

\subsection*{ 
  \gostTitleFont
  \redline
  1.1 Изучение структуры предприятия
} 

\subtitlespace

{\gostFont

  \par \redline ОАО «Савушкин Продукт» – это одно из ведущих предприятий пищевой промышленности Беларуси, специализирующееся на производстве молочной продукции. Компания выпускает широкий ассортимент изделий: молоко, кефир, творог, сметану, сыры и другие кисломолочные продукты. Будучи открытым акционерным обществом, предприятие поддерживает высокие стандарты корпоративного управления, что позволяет ему гибко реагировать на изменения рыночной среды и поддерживать устойчивое развитие. 

  \par \redline Начав с небольшого предприятия, «Савушкин Продукт» за годы работы трансформировался в современную производственную площадку с хорошо выстроенной системой управления. В основе успеха лежит последовательное внедрение инноваций, повышение качества продукции и постоянное совершенствование технологических процессов. Такие меры позволили не только укрепить позиции на внутреннем рынке, но и создать предпосылки для выхода на международную арену.

  \par \redline Продукция ОАО «Савушкин Продукт» представляет собой гармоничное сочетание традиционных методов молочной обработки и современных технологических решений. Такой подход обеспечивает изготовление высококачественных, безопасных для здоровья и вкусных продуктов, способных удовлетворить запросы самых требовательных потребителей. Дополнительно, постоянное обновление ассортимента через внедрение инноваций позволяет компании сохранять лидерские позиции в условиях жесткой конкуренции. 

  \par \redline ОАО «Савушкин Продукт» традиционно ассоциируется с высоким качеством молочной продукции и стабильными производственными процессами. Чтобы не только сохранять лидирующие позиции на отечественном рынке, но и задавать новые тренды в отрасли, предприятие создало Бюро перспективных разработок. Это специализированное подразделение выполняет важную стратегическую функцию, концентрируя усилия на поиске, разработке и внедрении инноваций. 

  \par \redline К основным задачам Бюро перспективных разработок относится разработка и поддержка иновационных инструментов для управления предприятием, разработка систем для создания описаний тех-процессов на предприятии, изучение новых технологий, которые могут повысить показатели производства, обеспечение бесперебойного функционирования предприятия. 
  
  \par \redline ОАО «Савушкин Продукт» имеет развитую сеть филиалов,и Бюро перспективных разработок занимается поддержанием работоспособности каждого из этих филиалов.
 

  \par \redline В заключении данной главы, стоит отметить, что ОАО «Савушкин Продукт» является лидером в выпуске молочной продукции в Республике Беларусь, и Бюро перспективных разработок внесло немалый вклад в это достяжение своей работой. Внедрение персективных разработок является важной деталью для достяжения таких вершин в наше время, также как и поодержание уже существующих технологий. ОАО «Савушкин Продукт» совмещает в себе традиционные методы производства и передовые технологии, что позволяет назвать данное предприятие передовым предприятием на територии Республики Беларусь. 

  \par
}

\subtitlespace

\subsection*{ 
  \gostTitleFont
  \redline
  1.2 Изучение должностных инструкций работников подразделения
} 

\subtitlespace

{\gostFont

  \par \redline Пастеризационная установка, она же пастеризационное-охладительная установка, предназначена непосредственно для самой пастеризации, т.е. для уничтожения вредных и опасных миркроорганизмов в молоке. Она состоит из множества различных элементов, изучая которые мы также сможем изучить принцип работы пастеризационной установки.  

  \par \redline Строго говоря, пастеризационная установка состоит из: 

  \par \redline 1. Балансного танка
  \par \redline 2. Подающего насоса
  \par \redline 3. Регулятора потока
  \par \redline 4. Секции регенеративного предварительного подогрева
  \par \redline 5. Центробежного очистителя
  \par \redline 6. Секции нагрева
  \par \redline 7. Трубы выдержки
  \par \redline 8. Вспомогательного насоса
  \par \redline 9. Системы нагрева горячей входы
  \par \redline 10. Секции регенеративного охлаждения
  \par \redline 11. Секции охлаждения
  \par \redline 12. Возвратного клапана
  \par \redline 13. Панели управления

  \par \redline И так, балансный танк является по сути резервуаром с молоком, оборудованным поплавковым входным клапаном, который регулирует расход молока и поддерживает его постоянный уровень в резервуаре. В балансном танке также имеется электрод минимального уровня. Он срабатывает, в том случае, когда уровень молока достигает минимальной точки, тем самым включая клапан распределения потока. Молоко заменяется водой и пастеризатор отключается. 

  \par \redline Подающий насос позволяет обеспечить молоком сам пастеризатор, выкачивая его из балансного танка. 

  \par \redline Для обеспечения устойчивого контроля температуры и постоянного времени выдержки пастеризационной установки используется регулятор потока. Он также необходим для поддержки расхода через пастеризатор на должно уровне.

  \par \redline Молоко, попав в секцию предварительного регенеративного подогрева, должно приобрести некоторую начальную температуру. Это осуществляется с помощью регенерированного тепла уже пастеризованного молока, которое в этот момент должно охлаждаться в секции регенеративного охлаждения. 

  \par \redline Секцию регенеративного нагрева можно разделить на три секции: секция начального регенеративного подогрева, секция предварительного очищения и секция конечного регенеративного подогрева. Такое разделение используется, поскольку существует необходимость в предварительной обработке молока между входной и выходной температурой молока в секции регенеративного подогрева. Секция начального регенеративного подогрева обычно придаёт молоку температуру около 55 $^{\circ}$С. После молоко отправляется в секцию предварительного очищения, представленную центробежным очистителем. После молоко поступает в секцию конечного регенеративного подогрева, где молоко добирает максимальную температуру. 

  \par \redline Эффективность такого нагревания молока регенеративным способом достигает 90-96\%, что отлично сказывается на энергосбережении. Остальную температуру входное молоко добирает за счёт горячей воды, температура которой на 2 или 3 $^{\circ}$С больше температуры пастеризации. Все перечисленные процессы происходят в секции подогрева. Горячая вода имеется благодаря системе нагрева воды.

  \par \redline Нагретое молоко попадает во внешнюю трубу выдержки, где температура проверяется датчиком. Этот датчик беспрерывно контактирует с регулятором температуры на панели управления, а также воздействует на регистрирующий прибор, который сохраняет температуру пастеризации. Тем самым происходит контроль процесса пастеризации.

  \par \redline Как только температура падает ниже минимума, активизируется возвратный клапан, позволяющий пастеризованному молоку попасть в секцию регенеративного охлаждения.
  
  \par \redline Секцию регенеративного охлаждения также можно разделить на две секции: первая секция регенеративного охлаждения и вторая секция регенеративного охлаждения. В первой секции регенеративного охлаждение горячее пастеризованное молоко передаёт свою температуру холодному, предварительно очищенному молоку, находящегося в секции регенеративного конечного подогрева молока. А во второй секции регенеративного охлаждения пастеризованное молоко отдаёт температуру необработанному молоку, находящемуся в секции регенеративного начального подогрева. 

  \par \redline После охлаждённое молоко попадает в секцию охлаждения, которую тоже можно разделить на две секции: первая секция охлаждения и вторая секция охлаждения. В первой секции пастеризованное молоко охлаждается холодной водой, а во второй секции молоко уже охлаждается ледёной водой. 

  \par 
}

\subtitlespace

\subsection*{  
  \gostTitleFont
  \redline
  1.3 Обзор задач, решаемых подразделением
} 

\subtitlespace

{\gostFont

  \par \redline На ОАО «Савушкин продукт», можно отметить следующее распределение задач и ответственности:
  \par \redline - Производство продукции: выпуск молочных товаров (йогурты, сыры, творог, молоко).
  \par \redline - Контроль качества и безопасность: соблюдение строгих стандартов.
  \par \redline - Логистика и снабжение: управление цепочками поставок сырья и готовой продукции.
  \par \redline - Инновации: внедрение новых технологий и разработка продуктов.
  \par \redline - Экология и устойчивое развитие: минимизация экологического следа.
  \par \redline - Социальная ответственность: поддержка местных сообществ и сотрудников.

  \par \redline Производственные задачи:
  \par \redline - Оптимизация производственных процессов. Подразделение решает задачи повышения эффективности линий, снижения энергозатрат и увеличения объемов выпуска. Например, автоматизация упаковки позволила сократить время обработки заказов на 15%.
  \par \redline - Расширение ассортимента. Разработка новых продуктов, таких как линейки органических йогуртов или безлактозных сыров, отвечает запросам рынка. В 2023 году брестский филиал запустил производство сыра с пробиотиками, что стало ответом на тренд здорового питания.
  \par \redline - Соблюдение стандартов. Производство соответствует требованиям ISO, ХАССП и белорусским ГОСТам, что критически важно для экспорта в ЕАЭС и другие страны.

  \par \redline Контроль качества и безопасность:
  \par \redline - Многоуровневый контроль. Каждая партия проходит лабораторные испытания на микробиологические и физико-химические показатели. Внедрена система отслеживания сырья от поставщика до прилавка.
  \par \redline - Работа с рекламациями. Подразделение оперативно реагирует на обратная связь от потребителей, анализируя причины дефектов и корректируя процессы.
  \par \redline - Сертификация. Поддержка международных сертификатов (например, Halal, Organic) расширяет возможности экспорта.

  \par \redline Логистика и снабжение:
  \par \redline - Управление поставками сырья. Сотрудничество с местными фермерскими хозяйствами обеспечивает стабильность поставок молока-сырья. Внедрена цифровая платформа для прогнозирования спроса.
  \par \redline - Дистрибуция готовой продукции. Оптимизация маршрутов доставки снизила логистические издержки на 10%. Подразделение взаимодействует с сетями супермаркетов в Беларуси, России, Казахстане.
  \par \redline - Складское хозяйство. Использование систем WMS (Warehouse Management System) повысило точность учета и скорость отгрузки.

  \par \redline Инновации и развитие:
  \par \redline - Внедрение технологий. На заводе в Бресте используются роботизированные линии розлива, IoT-датчики для мониторинга оборудования.
  \par \redline - Исследования и разработки. Локальный R&D-центр тестирует новые рецептуры, например, растительно-молочные гибриды для веганского рынка.
  \par \redline - Цифровизация. Внедрение ERP-системы интегрировало производство, логистику и финансы, сократив время согласований.

  \par \redline Экологическая ответственность:
  \par \redline - Снижение отходов. Переход на биоразлагаемую упаковку и переработка 95% производственных отходов.
  \par \redline - Энергоэффективность. Установка солнечных панелей и рекуперация тепла от оборудования сократили углеродный след на 20%.
  \par \redline - Водопользование. Система замкнутого цикла очистки воды уменьшила ее потребление на 30%.

  \par \redline Социальные инициативы:
  \par \redline - Поддержка сотрудников. Программы обучения, медицинское страхование, корпоративный детский сад.
  \par \redline - Участие в жизни региона. Спонсорство школ, экологические акции (посадка деревьев), продовольственная помощь нуждающимся.
  \par \redline - Развитие местной экономики. Закупки сырья у брестских фермеров создают рабочие места в регионе.

  \par \redline Брестское подразделение ОАО «Савушкин продукт» решает комплекс задач, направленных на укрепление позиций компании как локального и международного игрока. Через инновации, качество и ответственность подразделение вносит вклад в устойчивое развитие бизнеса и общества. Дальнейшие цели включают углубление цифровизации, расширение экспорта и усиление экологических инициатив, что соответствует глобальным трендам и стратегии компании.

  \par
}

\subtitlespace

\subsection*{
  \gostTitleFont
  \redline
  1.4 Обзор используемых технических средств на предприятии
} 

\subtitlespace

{\gostFont

  \par \redline На предприятии "Савушкин" в рамках автоматизации производственных процессов используется комплексная система автоматизации, состоящая из программного комплекса SCADA, программно-технического управляющего комплекса PTUSA и инструмента проектирования EasyEPLANner. Бюро перспективных разработок активно участвует в развитии и совершенствовании этих систем.

  \par \redline Рассмотрим аппаратное обеспечение, используемое на предприятии. Основу системы управления составляют программируемые логические контроллеры Phoenix Contact PLCnext серии AXC F 2152 и контроллеры WAGO PFC200, обеспечивающие гибкую платформу для реализации систем управления. Также используются промышленные компьютеры с операционными системами Linux для выполнения программы управления ptusa\_main и различные модули ввода-вывода для подключения датчиков и исполнительных механизмов.

  \par \redline Для контроля технологических процессов применяются различные датчики: температуры, давления, расходомеры для измерения объемов перекачиваемых продуктов, уровнемеры для контроля наполнения емкостей. Управление потоками осуществляется с помощью клапанов и задвижек с электроприводами, а для управления электродвигателями насосов и мешалок используются частотные преобразователи.

  \par \redline Сетевая инфраструктура предприятия включает промышленные коммутаторы для организации сети Ethernet, преобразователи интерфейсов для подключения устройств с различными протоколами связи и оптоволоконные линии связи для обеспечения надежной передачи данных между удаленными участками. Серверное оборудование представлено серверами для размещения SCADA-системы и баз данных, рабочими станциями операторов для мониторинга и управления технологическими процессами, а также системами резервного копирования для обеспечения сохранности данных.

  \par \redline Программное обеспечение предприятия включает несколько ключевых компонентов. Программный комплекс SCADA состоит из серверной части EasyServer для обработки и хранения данных, клиентской части Monitor для визуализации и управления технологическими процессами, редактора конфигурации базы данных DBXmlEditor и ряда вспомогательных библиотек: buglog.dll для ведения журнала ошибок, ConnectionLog.dll для регистрации подключений, MSUniServ.dll для универсального сервиса, ClientPLog.dll и ClientTLog.dll для ведения журналов, PGPropServ.dll и propservice.dll для работы с параметрами.

  \par \redline Система управления PTUSA включает основную программу управления ptusa\_main, непосредственно контролирующую технологические процессы, скрипты на языке Lua для реализации логики управления технологическими объектами и модульную архитектуру программного обеспечения, обеспечивающую гибкость и масштабируемость системы.

  \par \redline Важным компонентом системы автоматизации является EasyEPLANner – дополнение (Add-In) для EPLAN Electric P8 версии 2.9, представляющее собой инструмент для автоматизации проектирования систем управления. EasyEPLANner позволяет генерировать Lua-скрипты для программирования контроллеров Phoenix Contact PLCnext и WAGO PFC200, а также предоставляет средства для описания технологических объектов, операций, шагов и ограничений.

  \par \redline Для разработки компонентов системы используются различные средства: Delphi (версии 11.3 или 12.2) для разработки компонентов SCADA-системы, C\# для разработки EasyEPLANner, CMake для организации процесса сборки проекта PTUSA, Qt Creator и Microsoft Visual Studio для разработки и отладки программного обеспечения, PLCnext Technology C++ Toolchain для разработки под контроллеры Phoenix Contact. Система контроля версий Git используется для управления изменениями в программном коде, GitHub Actions – для непрерывной интеграции и автоматизированного тестирования, а SonarCloud – для анализа качества кода.

  \par \redline Для разработки SCADA применяются компоненты DevExpress (библиотека для создания пользовательского интерфейса), EControl SyntaxEditor (компонент для редактирования кода), Advantage Database Components (компоненты для работы с базами данных), oXml (библиотека для работы с XML), Embarcadero Sockets Components (компоненты для сетевого взаимодействия), Borland Database Engine (система управления базами данных) и QuickLogger (компонент для ведения журналов).

  \par \redline Тестирование системы осуществляется с помощью DUnit Tests для модульного тестирования компонентов SCADA, автоматизированного тестирования с использованием GitHub Actions для PTUSA и EasyEPLANner, а также анализа покрытия кода с помощью Codecov и SonarCloud для контроля качества тестирования.

  \par \redline Архитектура системы управления имеет многоуровневую структуру автоматизированной системы управления технологическими процессами, распределенную систему управления для повышения надежности и гибкости, а также интеграцию SCADA, PTUSA и EasyEPLANner для обеспечения комплексного подхода к автоматизации.

  \par \redline Принцип работы программы управления PTUSA основан на выполнении технологического процесса через последовательное выполнение операций для каждого технологического объекта, иерархической структуре управления (технологические объекты → операции → шаги → действия) и событийно-ориентированной модели для обработки команд и изменений режимов работы.

  \par \redline Взаимодействие компонентов SCADA построено на клиент-серверной архитектуре с центральным сервером EasyServer, системе мониторинга через клиентское приложение Monitor, хранении данных в реляционной базе данных с возможностью резервного копирования и обмене данными через специализированные библиотеки и сервисы.

  \par \redline EasyEPLANner играет важную роль в процессе проектирования, обеспечивая автоматизацию проектирования электрических схем в EPLAN Electric P8, описание технологических объектов и их свойств, генерацию Lua-кода для программирования контроллеров и сокращение трудозатрат инженеров-автоматизаторов и программистов.

  \par \redline Система автоматизации предприятия обладает широкими функциональными возможностями: визуализация технологических процессов в реальном времени, управление рецептурами и партиями продукции, система учета энергоресурсов для оптимизации энергопотребления, гибкая настройка алгоритмов управления с помощью скриптов Lua, ведение журналов и отчетов для документирования производственных процессов, автоматизированное проектирование с помощью EasyEPLANner.

  \par \redline Преимущества собственной разработки системы автоматизации включают адаптацию под специфические требования производства молочной продукции, оперативное внесение изменений в алгоритмы управления, независимость от сторонних разработчиков SCADA-систем, оптимизацию затрат на внедрение и поддержку системы автоматизации, возможность глубокой интеграции различных компонентов системы и открытый исходный код для EasyEPLANner, обеспечивающий прозрачность и возможность модификации.

  \par \redline Перспективы развития системы автоматизации предприятия включают расширение функциональности системы управления, интеграцию с системами машинного обучения для предиктивной аналитики, разработку цифровых двойников технологических процессов для моделирования и оптимизации, внедрение технологий промышленного интернета вещей (IIoT) для сбора и анализа данных, а также совершенствование EasyEPLANner для поддержки новых типов контроллеров и технологических объектов.

  \par \redline Внедрение и постоянное совершенствование собственной системы автоматизации, состоящей из программного комплекса SCADA, программно-технического управляющего комплекса PTUSA и инструмента проектирования EasyEPLANner, позволяет предприятию "Савушкин" обеспечивать высокий уровень автоматизации производственных процессов, гибко адаптировать систему под специфические требования производства и оперативно реагировать на изменения технологических процессов. Бюро перспективных разработок активно участвует в развитии системы, внедрении новых функциональных возможностей и повышении надежности работы всего комплекса.

  \par
}
