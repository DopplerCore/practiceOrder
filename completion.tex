\sectionbreak \section*{
	\gostTitleFont
	\redline
	ЗАКЛЮЧЕНИЕ
}

\subtitlespace

{\gostFont

	\par \redline В ходе прохождения преддипломной практики на предприятии ОАО «Савушкин продукт» в бюро перспективных разработок были успешно выполнены все поставленные задачи и получен ценный практический опыт.

	\par \redline В процессе практики были изучены:
	\begin{itemize}[leftmargin=2.15cm, labelwidth=0.65cm, labelsep=0.0cm] 

		\item[\theitemcntr.] Организационная и функциональная структура предприятия
		\addtocounter{itemcntr}{1}

		\item[\theitemcntr.] Должностные инструкции сотрудников бюро перспективных разработок
		\addtocounter{itemcntr}{1}

		\item[\theitemcntr.] Основные задачи и направления деятельности подразделения
		\addtocounter{itemcntr}{1}

		\item[\theitemcntr.] Технические средства и программное обеспечение, используемые на предприятии
		\addtocounter{itemcntr}{1}

		\setcounter{itemcntr}{1}
		\end{itemize}

		\par \redline В рамках практической работы были выполнены следующие задачи:
		\begin{itemize}[leftmargin=2.15cm, labelwidth=0.65cm, labelsep=0.0cm] 

		\item[\theitemcntr.] Сборка и запуск основных проектов предприятия: ptusa\_main, EasyEPLANner и SCADA.
		\addtocounter{itemcntr}{1}

		\item[\theitemcntr.] Изучение архитектуры и принципов работы программных комплексов.
		\addtocounter{itemcntr}{1}

		\item[\theitemcntr.] Доработка проекта SCADA в части документирования модулей BDE и последующее удаление неиспользуемых модулей, предложена система логирования.
		\addtocounter{itemcntr}{1}

		\item[\theitemcntr.] Для проекта ptusa\_main добавлен метод для виртуального пустого устройства STUB, который может быть использован заглушкой. 
		\addtocounter{itemcntr}{1}

		\item[\theitemcntr.] Разработка серверной части веб-версии SCADA на Java, используя Spring Web и веб-сокеты.
		\addtocounter{itemcntr}{1}

	\setcounter{itemcntr}{1}
	\end{itemize}

	\par \redline В процессе выполнения индивидуального задания был проведен анализ существующих платформ по распознованию медицинских патологий пациента, сформулированы требования к разрабатываемой системе и выбраны оптимальные технологии для ее реализации. Разработанная концепция веб-сервиса представляет собой комплексное решение, которое объединяет передовые технологии для автоматизации процесса анализа медицинских изображений. Он способен значительно улучшить качество диагностики, сократить время обработки данных и снизить нагрузку на медицинский персонал. Внедрение подобных систем в медицинскую практику открывает новые возможности для развития персонализированной медицины и улучшения качества жизни пациентов.

	\par \redline Практика позволила получить ценный опыт работы с современными технологиями веб-разработки, промышленными системами автоматизации и методами организации командной работы над крупными проектами. Полученные знания и навыки будут использованы при дальнейшей разработке дипломного проекта.

}

\setcounter{subchaptercntr}{1}
\setcounter{formulacntr}{1}
\setcounter{imagecntr}{1}