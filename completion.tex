\sectionbreak \section*{
\gostTitleFont
\redline
ЗАКЛЮЧЕНИЕ
}

\subtitlespace

{\gostFont
\par \redline В ходе прохождения преддипломной практики на предприятии ОАО «Савушкин продукт» в бюро перспективных разработок были успешно выполнены все поставленные задачи и получен ценный практический опыт.

\par \redline В процессе практики были изучены:
\begin{itemize}[leftmargin=2.15cm, labelwidth=0.65cm, labelsep=0.0cm] 

\item[\theitemcntr.] Организационная и функциональная структура предприятия
\addtocounter{itemcntr}{1}

\item[\theitemcntr.] Должностные инструкции сотрудников бюро перспективных разработок
\addtocounter{itemcntr}{1}

\item[\theitemcntr.] Основные задачи и направления деятельности подразделения
\addtocounter{itemcntr}{1}

\item[\theitemcntr.] Технические средства и программное обеспечение, используемые на предприятии
\addtocounter{itemcntr}{1}

\setcounter{itemcntr}{1}
\end{itemize}

\par \redline В рамках практической работы были выполнены следующие задачи:
\begin{itemize}[leftmargin=2.15cm, labelwidth=0.65cm, labelsep=0.0cm] 

\item[\theitemcntr.] Сборка и запуск основных проектов предприятия: ptusa\_main, EasyEPLANner и SCADA
\addtocounter{itemcntr}{1}

\item[\theitemcntr.] Изучение архитектуры и принципов работы программных комплексов
\addtocounter{itemcntr}{1}

\item[\theitemcntr.] Доработка проекта SCADA в части автозаполнения полей при импорте тегов OPC UA Browser
\addtocounter{itemcntr}{1}

\item[\theitemcntr.] Улучшение информативности сообщений об ошибках коммуникации в проекте ptusa\_main
\addtocounter{itemcntr}{1}

\item[\theitemcntr.] Разработка клиентской части веб-версии SCADA на React
\addtocounter{itemcntr}{1}

\setcounter{itemcntr}{1}
\end{itemize}

\par \redline В процессе выполнения индивидуального задания был проведен анализ существующих образовательных платформ, сформулированы требования к разрабатываемой системе и выбраны оптимальные технологии для ее реализации. Разработанная концепция образовательной платформы предполагает создание интерактивной системы для структурированного представления обучающих материалов с возможностью визуализации взаимосвязей между различными компонентами.

\par \redline Практика позволила получить ценный опыт работы с современными технологиями веб-разработки, промышленными системами автоматизации и методами организации командной работы над крупными проектами. Полученные знания и навыки будут использованы при дальнейшей разработке дипломного проекта.

}
