\sectionbreak \section*{ 
    \gostTitleFont
    \redline
    СПИСОК СОКРАЩЕНИЙ
}
\titlespace

{\gostFont

\begin{tabular}{p{0.85cm} p{1.75cm} p{0.3cm} p{131.5mm}}
    & НС & {--} & нейронная сеть. \\
    & Kaspersky MLAD & {--} & (Kaspersky Machine Learning for Anomaly Detection) методы машинного обучения кампании Kaspersky для выявления аномалий. \\
    & SCADA & {--} & (Supervisory for Control And Data Acquisition) система, обеспечивающая диспетчерское управление и сбор данных, относящаяся к классу программного обеспечения для создания автоматизированных систем управления технологическими процессами. \\
    & CNN & {--} & (Convolutional Neural Network) свёрточная нейронная сеть. \\
    & DenseNet & {--} & (Densely Connected Networks) тип сверточной нейронной сети, которая использует плотные связи между слоями через плотные блоки, где мы соединяем все слои напрямую друг с другом. \\
    & RNN & {--} & (Recurrent Neural Network) рекуррентная нейронная сеть. \\
    & TCN & {--} & (Temporal Convolutional Networks) темпоральная CNN. \\
    & MSE & {--} & (Mean Squared Error) среднеквадратичная ошибка. \\
    & SRN & {--} & (Simple Recurrent Network) простая реккурентная сеть. \\
    & LSTM & {--} & (Long Short-Term Memory) сеть долгой краткосрочной памяти. \\
    & GRU & {--} & (Gated Recurrent Units) управляемые рекуррентные нейроны. \\
    & MGU & {--} & (Minimal Recurrent Units) минимальные управляемые нейроны. \\
    & ПО & {--} & программное обеспечение. \\
    & ПП & {--} & программный продукт. \\
\end{tabular}

}
